%! TeX root = thesis.tex
\chapter{Test Objects}\label{app:model_table}

The following are the 3D models used to test the program.
The values listed under `source' have the following meanings:
\begin{itemize}
	\item \textbf{GC-PF}: Model downloaded from GrabCAD created by user paulocferreira 3D\cite{GC-PF}.
		User paulocferreira 3D numbers their basic models, which makes finding the source for a given model fairly straightforward.
		A specific model can be found by finding the model that corresponds to the number from the ID of a \textbf{GC-PF} sourced model: \verb|fcXXX|.
	% \item \textbf{GC-[A]}: Model was downloaded from GrabCAD, created by user XXX
	\item \textbf{10K}: Model downloaded from Thinki10K, a database of 3D models\cite{Thingi10K_paper, Thingi10K_app}.
		The models in Thingi10K were collected originally from Thingiverse, hence the name.
		To find the specific webpage for a given model, replace \verb|XXX| in \verb|https://ten-thousand-models.appspot.com/detail.html?file_id=XXX| with the ID from the table.
	\item \textbf{Thing}: Model downloaded from Thingiverse.
		Specific Thingiverse webpages can be found by replacing the \verb|XXX| in \verb|https://www.thingiverse.com/thing:XXX| with the ID from the table.
\end{itemize}

\csvreader[%
	longtable = |lcccl|,
	respect underscore = true,
	table head = \hline
	\textbf{ID} &
	\textbf{Source} &
	\textbf{Mesh} &
	\textbf{Class} &
	% \textbf{Explanation} &
	\textbf{Outcome}\\\hline\hline,
	late after line = \\\hline,
	late after last line = \\\hline
]{../resources/model_table.csv}{
	Filename=\id,
	% Description=\desc,
	Source=\src,
	Mesh=\mesh,
	Class=\class,
	% Reason=\reason,
	Outcome=\outcome
} {% Comment to prevent first couple lines from being shifted right slightly
	\id & \src & \mesh & \class & \outcome
}% https://tex.stackexchange.com/a/619232

