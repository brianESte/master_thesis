%! TeX root = thesis.tex
\chapter{Methodology}
\IMRADlabel{methods}

\section{Overview}
The basic premise is to segment the given model into primitive surfaces.
Each surface is unwrapped to a 2D representation, which undergoes a 2D segmentation to produce convex regions.
Upon each convex region a local path is planned.
The order in which each region is traversed by the robot is determined via a modified Traveling Salesman Problem.
Here, the start and end points of the salesman's path need not be the same.

\section{3D Segmentation}


\section{Watershed Segmentation}
-> Give overview of how WS segmentation works.

\subsection{Basic Procedure}
0. Apply height function at each vertex
1. Create initial mesh regions from local minima / minima plateaus
2. "Minima Expansion" Expand outward from each mesh region up to a certain depth, absorbing any regions encountered.
3. "Descent to Minima" From each un-indexed (check their word for this) vertex, follow the path of steepest descent until an indexed region is encountered.
All vertices traversed along this descent are added to the encountered region.

-> my modifications
\subsection{Mini-Merge}
As noted by Mangan and Whitaker(sp?) Descent to Minima Through testing and development it was observed that

\subsection{Boundary Smoothing}
Due to noise / other randomness in the mesh,


\section{Geometry Simplification}

