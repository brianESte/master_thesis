\chapter{Conclusion}
% This is what it /can/ do well, and /this/ is where it falls short.
This project, or at a minimum the overall concept chosen, proved more arduous than anticipated and requiring more time than available.
The complete pipeline is generally capable of processing models with planar surfaces.
At the time of writing, corners where four or more edges converge often cause issues, and should be avoided if possible.
% (? Given a few extra days, fixing this should be easily possible.)
As for curved surfaces, concave curves will at a minimum cause oversegmentation during the \textit{2D Segmentation} step, as discussed in the previous chapter.
The procedure is also unable to properly handle convex curves, but would require less effort to finish that functionality.
Furthermore, smooth transitions between primitive types constitute another open challenge.

Throughout development great emphasis was placed on reaching a point where the program and procedure could be tested in its entirety, and imroved from there.
In hindsight, this approach to development was ill-suited to a project of this nature, where the main building blocks are each complex and somewhat fragile.
If the time had been taken to test and refine \textit{Watershed Segmentation} and \textit{Surface Classification} to maturity, the issue regarding the height function selection could have been detected early enough to solve within the bounds of this project.

% Current conclusion(s):
% \begin{enumerate}
% 	\item I did not do sufficient research in the beginning, leading to wasted efforts later on
% 	\item My workflow was short-sighted, in that i only thought / considered a few steps ahead of my current position, rather than having a solid "big picture" view
% 	\item For simple flat objects with sharp edges, it works fine
% 	\item ...?
% \end{enumerate}

% I suggest /not/ continuing without serious reconsideration of either the objectives or the overall concept.
A continuation of this project without serious reevaluation of the overall concept is strongly disadvised.
The geometric primitive segmentation and classification concept is likely to be too rigid to effectively handle truly ``arbitrary'' geometry.
Should the range of the project's input geometry be restricted in future iterations, then this concept may prove viable.

\section{Outlook}
There are plenty of aspects of this project that could be improved.
These range from mere bug-fixes to replacing entire algorithms.
If the project were to be continued with all high-level components and steps, \textit{Interior Edge Extension} would require more testing to improve its robustness and fix minor bugs.

In order to maintain the current concept while allowing high-level and algorithm changes, the following improvements are recommended.

Further work would include:
\begin{itemize}
	\item An approach to handle both ``hard'' \textit{and} ``soft'' edges must be developed.
	\item Implement regression and classification functions for remaining geometric primitives
	\item \textit{Geometry Simplification} should be expanded and improved to solve the aforementioned shortcomings, and to improve robustness.
	\item The \textit{Interior Edge Extension} step should be replaced with one better suited to shapes with curved edges.
		The Boustrophedon Cellular Decomposition or an expansion thereof would provide a decent starting point for its replacement.
	\item The Traveling Salesman Problem should be implemented to join the individual cellular paths.
	% \item Inverse kinematics that better take into account the rotary table
	% \item Reachability was basically ignored in this project, so that should be explored...
	\item Reachability of the poses generated by the path planner will need to be taken into account before the procedure may be used outside of a test environment.
\end{itemize}

