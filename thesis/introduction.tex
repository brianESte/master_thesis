%! TeX root = thesis.tex
\chapter{Introduction}
\IMRADlabel{introduction}
Coverage path planning is the process of planning a path that guides an agent over the entirety of the region of interest.
% Parents would need an explainer for ``agent'', but the target audience should know already
% NOTE: (? could replace ROI with environment)
% The relevance of the research: How does this work fit into existing studies on the topic?
Applications thereof include floor cleaning robots~\cite{CCPP_cleaning_robots}, demining~\cite{CPP_demining}, automobile part spray painting~\cite{Automatic_spray_painting_path}, and processing of irradiated waste and debris~\cite{ROBBE}.
Without coverage path planning these would be at best less efficient, and at worst incomplete.
% The relevance of the research: How does this work fit into existing studies on the topic?
For the cleaning robot and demining applications the region of interest is a proper environment: a building's floor, and open terrain, respectively.
% In the case of floor cleaning robots and ocean mapping, the region of interest is the floor of a building, and the ocean floor, respectively.
Whereas for the car part spray painting and irradiated waste processing tasks the region of interest is an object, either in part or in whole.
This project follows the last example, in tackling the processing of hazardous waste, specifically low and medium level nuclear waste created during dismantling.
% This project tackles the same application as the last example, the processing of hazardous waste.

% The topic, in context: what does the reader need to know to understand the thesis?
% Throughout the operation of a nuclear power plant, various components become contaminated with nuclear material and irradiated.
Throughout normal operation at nuclear power plants, various components come into contact with radioactive material and become contaminated.
During the decommissioning and dismantling of these plants, the irradiated components pose a waste disposal challenge.
Parts within the nuclear power plant that are expected to be subjected to nuclear contamination receive a protective coating during installation~\cite{NRC_coatings}.
By absorbing radiation and blocking radioactive nuclides from reaching the metal underneath, the coatings protect metal components from corrosion, and reduce general wear.
Decommissioning is also facilitated by the coating, because it allows the underlying metal to be reprocessed and recycled like ordinary waste, once the coating has been completely removed~\cite{DeconTechInDecommissioning}.
This greatly reduces the volume of material that requires special treatment and storage.

Manually cleaning irradiated waste and debris risks contamination to the operators and is often physically taxing.
By creating a system that can automate the cleaning process this project hopes to reduce the risk to human operators.
The envisioned setup consists of a robotic arm with an end effector mounted ablation laser to handle coating removal.
Contaminated parts would be placed in a vice on a rotary table within the robotic arm's workspace.
% For the purpose of removing the contaminated coatings a combination ablation laser and vacuum aparatus is envisioned.

% Focus and scope: What specific aspect of the topic will be addressed?
Due to the destructive nature of the dismantling process, the geometry of the objects to be processed is unknown, and assumed to be arbitrary.
Upstream of this project the target objects are scanned and a 3D mesh representation created.
% In order to plan a coverage path on arbitrary geometry
The inputs to the program developed during this project are the target object's mesh, and the ablation laser's characteristics.

The objective of this project is to develop a coverage path planning process capable of handling arbitrary geometry.
% This work draws on the fields of geometric curvature, mesh segmentation, geometric shape fitting, high-level path planning, and low-level path planning.
To achieve this, the fields of geometric curvature, mesh segmentation, geometric shape fitting, high-level path planning, and low-level path planning are drawn upon.
The concept proposed in this work is one of segmentation and simplification.
The object's mesh is segmented along regions of high geomtric curvature, in order to produce mesh sections that may be classified as a certain geometric primitive.
% The 3D segmentation process is performed using Watershed segmentation.
Simplified geometric representations are then created from the classified mesh sections.
Non-convex surfaces are segmented further to yield convex surfaces.
Next a basic back-and-forth path is planned on each convex region.
These individual paths are unified via a Traveling Salesman Problem and then sent to a robotic arm for demonstration.
The procedure developed is assessed based on the successful path planning of various CAD objects.

% To achieve this, research from various fields (geometric curvature, mesh segmentation, geometric shape fitting, UAV path planning, cellular path planning...) is brought together.

% An overview of the paper structure: what does each section contribute to the overall aim?
The next chapter discusses research related to the project.
Chapter~\ref{sec:bkgd} lays the foundation upon which this work is based.
The methods developed throughout this project are explained in Chapter~\ref{sec:mtdy}, and the results are presented in Chapter~\ref{sec:eval}.
A discussion of the issues encountered as well as possible improvements is found in Chapter~\ref{sec:discussion}.
This paper concludes with suggestions regarding the future of this project.
% NOTE: still not quite satisfied with this line ^

