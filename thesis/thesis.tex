% !TeX program = lualatex
% TeX TXS-program:compile = txs:///lualatex/[-shell-escape]
%% Document based on `DEMO-TUDaThesis.tex' version 3.28 (2022/11/15),
%% A part of
%% TUDa-CI -- Corporate Design for TU Darmstadt
%%

\documentclass[
	ngerman,
	ruledheaders=section,%Ebene bis zu der die Überschriften mit Linien abgetrennt werden, vgl. DEMO-TUDaPub
	class=report,% Basisdokumentenklasse. Wählt die Korrespondierende KOMA-Script Klasse
	thesis={type=master},% Dokumententyp Thesis, für Dissertationen siehe die Demo-Datei DEMO-TUDaPhd
	accentcolor=9c,% Auswahl der Akzentfarbe
	custommargins=true,% Ränder werden mithilfe von typearea automatisch berechnet
	marginpar=false,% Kopfzeile und Fußzeile erstrecken sich nicht über die Randnotizspalte
	%BCOR=5mm,%Bindekorrektur, falls notwendig
	parskip=half-,%Absatzkennzeichnung durch Abstand vgl. KOMA-Script
	fontsize=11pt,%Basisschriftgröße laut Corporate Design ist mit 9pt häufig zu klein
%	logofile=example-image, %Falls die Logo Dateien nicht vorliegen
]{tudapub}


% Der folgende Block ist nur bei pdfTeX auf Versionen vor April 2018 notwendig
\usepackage{iftex}
\ifPDFTeX
	\usepackage[utf8]{inputenc}%kompatibilität mit TeX Versionen vor April 2018
\fi
% For dealing with SVG issues
% \usepackage{ifluatex}
% \ifluatex
% 	\usepackage{pdftexcmds}
% 	\makeatletter
% 	\let\pdfstrcmp\pdf@strcmp
% 	\let\pdffilemoddate\pdf@filemoddate
% 	\makeatother
% \fi
% \usepackage{svg}

\usepackage{standalone}
\usepackage{tikz}
\usetikzlibrary{
	arrows.meta,
	calc,
	fit,
	matrix,
	positioning,
	ext.paths.ortho
}
% Define global styles
\tikzset{
	FC-Node/.style={rectangle,semithick,align=center,rounded corners=5pt,draw},
	FC-Arrow/.style={-{Stealth[round]},semithick,rounded corners=4pt}
}
\def \globalscale {1.000000}
% TODO: integrate the color into tikz
\definecolor{c179c7d}{RGB}{23,156,125}
% for multifigure environments
\usepackage{subcaption}
% \usepackage{svg}

% Pseudocode
\usepackage{algpseudocode}
\usepackage{algorithm}

%%%%%%%%%%%%%%%%%%%
%Sprachanpassung & Verbesserte Trennregeln
%%%%%%%%%%%%%%%%%%%
\usepackage[main=english, ngerman]{babel}
\usepackage[autostyle]{csquotes}% Anführungszeichen vereinfacht

% Falls mit pdflatex kompiliert wird, wird microtype automatisch geladen, in diesem Fall muss diese Zeile entfernt werden, und falls weiter Optionen hinzugefügt werden sollen, muss dies über
% \PassOptionsToPackage{Optionen}{microtype}
% vor \documentclass hinzugefügt werden.
\usepackage{microtype}

%%%%%%%%%%%%%%%%%%%
%Literaturverzeichnis
%%%%%%%%%%%%%%%%%%%
\usepackage[
	backend=bibtex,
	% Year (descending) -> Name -> Title
	sorting=ydnt
]{biblatex}   % Literaturverzeichnis
\addbibresource{references.bib}
%\bibliography{references}

%%%%%%%%%%%%%%%%%%%
%Paketvorschläge Tabellen
%%%%%%%%%%%%%%%%%%%
%\usepackage{array}     % Basispaket für Tabellenkonfiguration, wird von den folgenden automatisch geladen
\usepackage{tabularx}   % Tabellen, die sich automatisch der Breite anpassen
\usepackage{longtable} % Mehrseitige Tabellen
%\usepackage{xltabular} % Mehrseitige Tabellen mit anpassbarer Breite
\usepackage{booktabs}   % Verbesserte Möglichkeiten für Tabellenlayout über horizontale Linien
\usepackage{csvsimple}
% ^ to parse CSV for tables

%%%%%%%%%%%%%%%%%%%
%Paketvorschläge Mathematik
%%%%%%%%%%%%%%%%%%%
\usepackage{mathtools} % erweiterte Fassung von amsmath
%\usepackage{amssymb}   % erweiterter Zeichensatz
%\usepackage{siunitx}   % Einheiten

%Formatierungen für Beispiele in diesem Dokument. Im Allgemeinen nicht notwendig!
\let\file\texttt
\let\code\texttt
\let\tbs\textbackslash
\let\pck\textsf
\let\cls\textsf

\usepackage{pifont}% Zapf-Dingbats Symbole
\newcommand*{\FeatureTrue}{\ding{52}}
\newcommand*{\FeatureFalse}{\ding{56}}

% \addTitleBoxLogo*{\includegraphics[width=0.65\linewidth]{../resources/logos/CCPS_logo.pdf}}
\addTitleBoxLogo*{\includegraphics[width=\linewidth]{../resources/logos/CCPS_logo.pdf}}
\addTitleBox{
%	\includegraphics[width=\linewidth]{../resources/logos/igdLogo.png}
	\resizebox{\linewidth}{!}{
		\includestandalone{../resources/logos/igd_logo}
	}
}
%\addTitleBoxLogo*{\includesvg[width=\linewidth]{igd_logo}}

\begin{document}

\Metadata{
	title=Coverage Path Planning on Arbitrary via Surface Segmentation and Simplification,
	author=Brian Stephenson
}

\title{Coverage Path Planning on Arbitrary Geometry via Surface Segmentation and Simplification}
% \subtitle{\LaTeX{} using TU Darmstadt's Corporate Design}
\subtitle{\textmd{Studiengang Elektrotechnik und Informationstechnik}}
\author[B. Stephenson]{Brian Stephenson}%optionales Argument ist die Signatur,
\reviewer*[Examiner,Advisor,Advisor]{Prof. Dr.-Ing. Rolf Findeisen \and Dr.-Ing Eric Lenz \and MSc. Hasan Kutlu}%Gutachter
% \reviewer[Advisor]{Dr.-Ing Eric Lenz \and MSc. Hasan Kutlu}%Gutachter

%Diese Felder werden untereinander auf der Titelseite platziert.
%\department ist eine notwendige Angabe, siehe auch dem Abschnitt `Abweichung von den Vorgaben für die Titelseite'
\department{metro} % Das Kürzel wird automatisch ersetzt und als Studienfach gewählt, siehe Liste der Kürzel im Dokument. -> metro
\institute{Mechatronics}
\group{Fraunhofer IGD}

\submissiondate{\today}
\examdate{22.11.25}

% Hinweis zur Lizenz:
% TUDa-CI verwendet momentan die Lizenz CC BY-NC-ND 2.0 DE als Voreinstellung.
% Die TU Darmstadt hat jedoch die Empfehlung von dieser auf die liberalere
% CC BY 4.0 geändert. Diese erlaubt eine Verwendung bearbeiteter Versionen und
% die kommerzielle Nutzung.
% TUDa-CI wird im nächsten größeren Release ebenfalls diese Anpassung vornehmen.
% Aus diesem Grund wird empfohlen die Lizenz manuell auszuwählen.
%\tuprints{urn=XXXXX,printid=XXXX,year=2022,license=cc-by-4.0}
% To see further information on the license option in English, remove the license= key and pay attention to the warning & help message.

% \dedication{Für alle, die \TeX{} nutzen.}

\maketitle

\affidavit% oder \affidavit[digital] falls eine rein digitale Abgabe vorgesehen ist.
% Es gibt mit Version 3.20 die Möglichkeit ein Bild als Signatur einzubinden.
% TUDa-CI kann nicht garantieren, dass dies zulässig ist oder eine eigenhändige Unterschrift ersetzt.
% Dies ist durch Studierende vor der Verwendung abzuklären.
% Die Verwendung funktioniert so:
%\affidavit[signature-image={\includegraphics[width=\width,height=1cm]{example-image}}, <hier können andere Optionen wie z.B. affidavit=digital zusätzlich stehen>]

\tableofcontents

%! TeX root = thesis.tex
\chapter{Introduction}
\IMRADlabel{introduction}
Coverage path planning is the process of planning a path that guides an agent over the entirety of the region of interest.
% The relevance of the research: How does this work fit into existing studies on the topic?
Applications thereof include floor cleaning robots~\cite{CCPP_cleaning_robots}, demining~\cite{CPP_demining}, automobile part spray painting~\cite{Automatic_spray_painting_path}, and processing of irradiated waste and debris~\cite{ROBBE}.
Without coverage path planning these would be at best less efficient, and at worst incomplete.
% The relevance of the research: How does this work fit into existing studies on the topic?
For the cleaning robot and demining applications the region of interest is a proper environment: a building's floor, and open terrain, respectively.
% In the case of floor cleaning robots and ocean mapping, the region of interest is the floor of a building, and the ocean floor, respectively.
Whereas for the car part spray painting and irradiated waste processing tasks the region of interest is an object, either in part or in whole.
This project follows the last example, in tackling the processing of hazardous waste, specifically low and medium level nuclear waste created during dismantling.
% This project tackles the same application as the last example, the processing of hazardous waste.

% The topic, in context: what does the reader need to know to understand the thesis?
% Throughout the operation of a nuclear power plant, various components become contaminated with nuclear material and irradiated.
Throughout normal operation at nuclear power plants, various components come into contact with radioactive material and become contaminated.
During the decommissioning and dismantling of these plants, the irradiated components pose a waste disposal challenge.
Parts within the nuclear power plant that are expected to be subjected to nuclear contamination receive a protective coating during installation~\cite{NRC_coatings}.
By absorbing radiation and blocking radioactive nuclides from reaching the metal underneath, the coatings protect metal components from corrosion, and reduce general wear.
Decommissioning is also facilitated by the coating, because it allows the underlying metal to be reprocessed and recycled like ordinary waste, once the coating has been completely removed~\cite{DeconTechInDecommissioning}.
This greatly reduces the volume of material that requires special treatment and storage.

Manually cleaning irradiated waste and debris risks contamination to the operators and is often physically taxing.
By creating a system that can automate the cleaning process this project hopes to reduce the risk to human operators.
The envisioned setup consists of a robotic arm with an end effector mounted ablation laser to handle coating removal.
Contaminated parts would be placed in a vice on a rotary table within the robotic arm's workspace.
% For the purpose of removing the contaminated coatings a combination ablation laser and vacuum aparatus is envisioned.

% Focus and scope: What specific aspect of the topic will be addressed?
Due to the destructive nature of the dismantling process, the geometry of the objects to be processed is unknown, and assumed to be arbitrary.
Upstream of this project the target objects are scanned and a 3D mesh representation created.
% In order to plan a coverage path on arbitrary geometry
The inputs to the program developed during this project are the target object's mesh, and the ablation laser's characteristics.

The objective of this project is to develop a coverage path planning process capable of handling arbitrary geometry.
% This work draws on the fields of geometric curvature, mesh segmentation, geometric shape fitting, high-level path planning, and low-level path planning.
To achieve this, the fields of geometric curvature, mesh segmentation, geometric shape fitting, high-level path planning, and low-level path planning are drawn upon.
The concept proposed in this work is one of segmentation and simplification.
The object's mesh is segmented along regions of high geometric curvature, in order to produce mesh sections that may be classified as a certain geometric primitive.
% The 3D segmentation process is performed using Watershed segmentation.
Simplified geometric representations are then created from the classified mesh sections.
Non-convex surfaces are segmented further to yield convex surfaces.
Next a basic back-and-forth path is planned on each convex region.
These individual paths are unified via a Traveling Salesman Problem and then sent to a robotic arm for demonstration.
The procedure developed is assessed based on the successful path planning of various CAD objects.

% To achieve this, research from various fields (geometric curvature, mesh segmentation, geometric shape fitting, UAV path planning, cellular path planning...) is brought together.

% An overview of the paper structure: what does each section contribute to the overall aim?
The next chapter discusses research related to the project.
Chapter~\ref{sec:bkgd} lays the foundation upon which this work is based.
The methods developed throughout this project are explained in Chapter~\ref{sec:mtdy}, and the results are presented in Chapter~\ref{sec:eval}.
A discussion of the issues encountered as well as possible improvements is found in Chapter~\ref{sec:discussion}.
This paper concludes with suggestions regarding the future of this project.
% NOTE: still not quite satisfied with this line ^



%! TeX root = thesis.tex
\chapter{Related Work}

The main sub-topics within path planning are pathfinding and coverage path planning.
The goal of pathfinding is to determine the shortest path between two points while avoiding any obstacles.
Pathfinding has applications ranging from navigation to network routing.
% Developed in 1959, Dijsktras alg is among the oldest and most well known pathfinding algorithms.
% It determines the shortest path between two nodes in a weighted graph by visiting the nearest unexplored node and keeping track of the shortest path from the starting node to the other nodes.
The primary pathfinding algorithms are Dijkstra's algorithm~\cite{Dijkstra, Improved_Dijkstra} which determines the shortest path between two nodes in a weighted graph, and A* (A-Star)\cite{A_Star_lit_review, A_Star_beginners, A_Star_in_computer_games} which is effectively Dijkstra's algorithm with a heuristic function applied.
Dijkstra's algorithm is well suited for navigating between cities~\cite{Dijkstra_for_railroads}, because a network of connected cities is effectively a graph with edges weighted by distance.
While Dijkstra's algorithm picks the next node to check only considering the cost from the start node to the next node, A* also factors in an estimate of the cost from that node to the goal.

In contrast to this, the goal of coverage path planning (CPP) is to traverse as much of the given working area as possible.
The objective of CPP is typically for an agent to perform some function over the entire working area.
Examples of this include spray painting car parts~\cite{Automatic_spray_painting_path}, spray forming glass reinforced cement~\cite{Robotic_grc_spraying}, metal polishing~\cite{Metal_polishing_robot_sys}, floor cleaning~\cite{CCPP_guidance_for_cleaning_robots}, demining~\cite{CPP_demining}, lawn mowing~\cite{CPP_autonomous_lawn_mower}, farming~\cite{Vision_perception_auto_harvester, CPP_alg_agriculture}, underwater inspection of ship hulls~\cite{CPP_inspect_complex_structures}, and producing mosaicked images of the ocean floor~\cite{Terrain_covering_AUV}.

In the past few years drones have become increasing popular, including as CPP agents~\cite{CPP_UAV_survey, CPP_2D_convex_regions_uav, CPP_multi_UAV, CPP_spraying_drones}.
While drones present a high degree of translational maneuverability, their orientation relative to gravity remains fairly constant, which can limit their applications.
Luna et al.\ presented a multi-UAV approach to scan a set of disjointed land areas~\cite{CPP_multi_UAV}.
Their approach is hierarchical: A global plan is developed and sent to each UAV, which then refines its part of the plan.

CPP algorithms can be off-line, in which the environment is known at the start and is assumed constant, or on-line, in which the path planning agent must deal with an unknown or changing environment~\cite{CPP_survey_for_robotics}.
On-line applications include the aforementioned car part spray painting, panel spray forming~\cite{Robotic_grc_spraying}, and metal polishing.
Liu et al.\ propose an algorithm for floor cleaning robots in an unknown environment by combining random path movement and locally complete coverage paths~\cite{CCPP_cleaning_robots}.
Their locally complete coverage paths follow a ``comb-like'' path that likens a Boustrophedon path.

% \section{Spatial Complexity}
The spatial complexity of CPP problems extends to 2.5D and 3D, though solutions to such problems typically involve segmenting the working region into sub-regions that can be approximated as 2D without significant loss of precision.

2D applications of coverage path planning are characterized by the region of interest (ROI) and the agent's range of motion being limited to 2D.
Any 3rd dimension aspects are negligible.
The classic example thereof is robot vacuum cleaners, such as iRobot's ``Roomba'' and Electrolux's ``Trilobite''\cite{CCPP_cleaning_robots}, whereas less obvious examples include drones that operate at a constant elevation~\cite{CPP_2D_convex_regions_uav}.
Gajjar et al.\ discretize the given 2D space to a grid in order to simplify agent movement~\cite{CCPP_known_2D_env}.

2.5D problems exhibit a working area with non-uniform height, but can otherwise still be mapped to a 2D space.
Jin and Tang developed a CPP algorithm for uneven farmland terrain which accounted for farmland relevant costs, such as headland turning, soil erosion, and curved paths~\cite{CPP_farming_terrain}.
Hameed et al.\ calculate the height value for their waypoints via bilinear interpolation from the Digital Elevation Model of their target area~\cite{CPP_2.5D_agriculture}.
Gao et al.\ use a 2.5D grid discretization as the basis for their algorithm~\cite{CPP_2.5D_grid_map}.
Coverage path planning performed by UAVs in a 2D space covering an otherwise 3D ROI is seemingly common~\cite{CPP_2.5D_SAA_grid_based_UAV_3D_recons, CPP_2.5D_UAV_3D_terrain_recons, CPP_multi_UAV, CPP_2D_convex_regions_uav}.
% This style of CPP feels more 2D than 2.5D, despite the results typically being 3D
Galceran and Carreras demonstrate the complexity of a ROI with varying distance to the agent~\cite{CPP_2.5D_seabed_2012}, as holding the depth of their autonomous underwater vehicle constant results in a varying field of view (FOV) observed.
% NOTE: If i talked more about AUVs, it would be worth creating and using the initialism...
They solve this by segmenting the given seabed map into regions of depth constant enough to be approximated as 2D.
In a subsequent paper Galceran and Carreras tackle a similar problem by segmenting the given bathymetric map into ``high-slope'' regions and relatively planar regions~\cite{CPP_2.5D_seabed_2013}.

Three dimensional CPP problems require movement in all 3 dimensions relative to the ROI~\cite{CPP_survey_for_robotics}.
Cao et al.\ present a hierarchical solution to CPP in thoroughly 3D environments, by dividing the ROI into subspaces and planning paths for the individual subspaces~\cite{HiCPP_cplx_3D_env}.
They rely on a global Traveling Salesman Problem (TSP) to determine the traversal order of the subspaces, as well as local TSPs to plan the traversal of viewpoints within each subspace.

The ROIs in most of the CPP applications presented above are environments, as opposed to planning a path on an object.
In such situations the agent is typically a robotic arm with sensors and/or tools affixed to the end effector~\cite{Metal_polishing_robot_sys, Automatic_spray_painting_unknown_parts}.
The target of Asakawa and Takeuchi's path planning algorithm was a 6 degree of freedom (DOF) robotic arm with end effector mounted spray painting tool~\cite{Automatic_spray_painting_path}.
They developed an algorithm to generate robot control commands from CAD data for the purpose of replacing the manual spray painting of car bumpers.
Atkar et al.\ also examine trajectory generation for a spray painting robot~\cite{Uniform_cov_auto_surfaces, Exact_cell_decomp_orientable_surfaces}.
They follow the common approach of segmenting the target surface into simple regions to be handled individually.
their approach to local path planning is based on optimizing the seed curve to ensure uniform paint deposition and minimize the overall cycle time.
Seed curve in this context is the curve that sets the orientation of the back-and-forth motions used to cover the region.
% NOTE: I do not fully understand Atkar et al.'s "virtual surfaces", so mention thereof has been removed.
% The algorithm proposed in their earlier work does not plan a path on the object's surface directly, but rather a virtual surface offset from the actual surface.
% The path is formed by the repeated intersection of a slicing plane with the offset surface.

% Mizugaki et al.\ created a program to generate fractal paths so that a 6-DOF industrial robot

% \section{3D Segmentation}
Decomposing a fully 3D space into more manageable subspaces is typically done either in a hierarchical manner~\cite{HiCPP_cplx_3D_env}, or via segmentation of the target's mesh representation~\cite{Mesh_segm_technik_survey}.
% TODO:?
Most mesh segmentation methods use a heuristic to determine which vertices belong to which regions.
These heuristics are generally based on mesh attributes, such as differences in adjacent vertex normals, dihedral angles, curvature, symmetry, and convexity, among others.
% 1. Region growing
the simplest segmentation method examined by Shamir is known as region growing, in which sub-meshes start from a seed element and expand therefrom~\cite{Mesh_segm_technik_survey}.
The next segmentation method is ``multiple source region grow'', where multiple seeds are picked and expanded in parallel.
Another approach was ``iterative clustering'', which sought to optimally partition the given mesh into $n$ sub-meshes, where $n$ was a set input parameter.
This method follows the classic $k$-means clustering algorithm, but applied to vertices in a mesh.

% (? I had considered writing more about the different types of 3D mesh decomposition, but that would have been another page, at least, and the chapter is already ~3.5 pages...)
% TODO: consider expanding on the types of mesh segmentation discussed in MSTS

Pichler et al.\ present 3D segmentation method based not on mesh attributes, but on physical feature detection.
Their procedure compares a point cloud sampled from an unknown industrial component with a ``Geometry Library'', a collection of complex geometric definitions~\cite{Automatic_spray_painting_unknown_parts}.
In the paper presented, their geometry library was only capable of detecting ribs and cavities.
The concept was to detect elementary geometries that could be linked to a specific predefined path planning procedure.
Non-rib / -cavity regions are termed ``free-form'', and are painted with a pattern of straight strokes.

% \section{2D Segmentation}
Given a complex planar representation of a ROI most path planning algorithms will first decompose the complex map into simpler connected regions~\cite{Cell_decomposition_survey, GdR}.
The majority of these methods exist for the purpose of pathfinding, and are not well suited for coverage path planning.
The following are decomposition methods applicable to coverage path planning.

% \item Grid decomposition: in which the configuration space is decomposed into a grid with cells marked as ``empty'' or ``full'' depending on whether the cell is free of obstacles or not.
The \textit{exact cellular decomposition} decomposes a map with polygonal obstacles into trapezoids and triangles by sweeping a vertical line across the map, creating a new cell at each obstacle vertex~\cite{Robot_motion_planning}.
A graph is then created from the cells with adjacent cells connected in the graph.
For pathfinding an algorithm such as Dijkstra's is applied to the graph to determine the cell traversal order in order for an agent to reach a given destination.
For coverage path planning a local path is planned for each cell, and an algorithm such as the Traveling Salesman Problem would be applied to the graph to join the local paths together and ensure each cell is visited.

Windmill decomposition is also based on a map with polygonal obstacles, but the decomposition is created by extending a line in the same rotational direction from each obstacle edge~\cite{Windmill_decomp_for_free_pp}.
Hence, given an unrotated rectangular obstacle, a line would be extended from the top edge to the right, the left edge upwards, bottom edge to the left, and right edge downwards.
The extended lines end when they encounter another object, the map boundary or another line extension.
From there the procedure follows that of \textit{exact cellular decomposition}.

Choset and Pignon proposed the boustrophedon cellular decomposition (BCD)~\cite{Bous_cellular_decomp}.
Based on the \textit{exact cellular decomposition}, a vertical line is swept across the polygonal map.
Cells are created only at the horizontal ends of obstacles, that is, the left-most and right-most points.
In their paper, they term these points ``IN'' events and ``OUT'' events, presumably when an obstacle moves \textit{in}to or \textit{out} of the swept line.
This adaptation has the effect of merging all cells that would have been formed via \textit{exact cellular decomposition} between the ``IN'' and ``OUT'' events, thus reducing the number of cells that need to be traversed.

Nielsen et al.\ present an algorithm to minimize the number of turns by segmenting the given region into \textit{optimal} convex sub-polygons~\cite{IntEdgeExt}.
In order to apply a square zig-zag pattern akin to the boustrophedon path, they divide the given map into convex sub-polygons by extending the map's interior edges, hence their name for the algorithm \textit{interior extension of edges}.
% They named their algorithm \textit{interior extension of edges} because it forms sub-polygons by extending the interior edges of the given map.
In order to minimize the number of turns within a given sub-polygon cell, they propose merging the sub-polygon cells into \textit{optimal} sub-polygons.
They term a convex polygon composed from one or more adjacent sub-polygon cells a \textit{convex merge option}.
The set of convex merge options is created from the initial sub-polygons cells.
To determine which subset of convex merge options constitutes the \textit{optimal} sub-polygons is a set cover problem with the optimal sub-polygons selected via an integer programming model.

% To plan a path within a convex or semi-convex (as is the case via BCD) polygon two motion planning options are immediately apparent: a spiral patter and a back-and-forth pattern also known as a boustrophedon path.
% Between these, the boustrophedon path is generally favored over the spiral pattern because it is simpler and because the spiral pattern requires a direction change at each corner, whereas an omnidirectional agent, such as a UAV or robotic arm, following a boustrophedon path need not change orientation between sweep lines, only position.

Based on the complexity of the problems and solutions discussed above, the task at hand is neither small nor simple.
This project examines the problem of planning a coverage path on an object of unknown and arbitrary geometry via segmentation, simplification, and if necessary further segmentation.
% Is it really necessary to give a statement on the subsequent chapters and their content?
% The initial segmentation step is descibed in Section 3.1 (replace with ref)
% The initial segmentation step, subsequent simplification, and potential further segmentation are explained in Sections 4.2, 4.3, and 4.4, respectively.



%! TeX root = thesis.tex
\chapter{Background}\label{sec:bkgd}
\tikzset{HighLevelProcedure/.pic={
	\def\dx{6mm}
	% nodes
	\node[] (Mesh) {Mesh};
	\node[FC-Node] (3DSeg) [right=\dx of Mesh] {3D Segmentation};
	\node[FC-Node] (GeoSimp) [right=\dx of 3DSeg] {Geometry Simplification};
	\node[FC-Node] (2DSeg) [right=\dx of GeoSimp] {2D Segmentation};
	\node[FC-Node,text width=27mm] (Bpath) [below=10mm of Mesh, xshift=10mm] {Cellular Path Planning};
	\node[FC-Node,text width=30mm] (TSP) [right=\dx of Bpath] {Modified TSP};
	\node[FC-Node,text width=20mm] (InvKin) [right=\dx of TSP] {Inverse Kinematics};
	\node[text width=20mm] (Poses) [right=\dx of InvKin] {Actuator Poses};
	% connections
	\draw [FC-Arrow] (Mesh) -- (3DSeg);
	\draw [FC-Arrow] (3DSeg) -- (GeoSimp);
	\draw [FC-Arrow] (GeoSimp) -- (2DSeg);
	\draw [FC-Arrow, rounded corners=5pt] (2DSeg.south) |-| (Bpath.north);
	\draw [FC-Arrow] (Bpath) -- (TSP);
	\draw [FC-Arrow] (TSP) -- (InvKin);
	\draw [FC-Arrow] (InvKin) -- (Poses);
}}

\begin{figure}[hb]
	\centering
\begin{tikzpicture}
	\pic at (0,0) {HighLevelProcedure};
\end{tikzpicture}
	\caption{Graphical overview of the path planning and execution procedure}
	\label{fig:bkgd_overview}
\end{figure}
The objective is to plan a path covering the surface of an unknown object.
The intended application of this is to clean the surface of the object via laser ablation.
% using a planar laser.
The general procedure is described below and presented visually in Figure~\ref{fig:bkgd_overview}.
The 3D triangular mesh of an object is provided as input.
First the mesh is segmented and its mesh sections classified in the \textit{3D Segmentation} step.
In \textit{Geometry Simplification} idealized surfaces are created from each classified mesh section.
\textit{2D Segmentation} checks the convexity of each idealized surface's unwrapped form, and segments the non-convex ones into convex polygons.
The actual path planning takes place in \textit{Cellular Path Planning}, where a simple back-and-forth path is planned on each convex surface.
The task of combining the individual paths into a single path is a form of Traveling Salesman Problem (TSP) and is handled in \textit{Modified TSP}.
Once the path is complete, \textit{Inverse Kinematics} handles sending the path's waypoints to the robot arm tasked with carrying out the cleaning operation.

\vspace{1cm} % To force notation section to start on a new page
\section{A Note on Notation}
To transform a point from one coordinate system to another one typically uses a transformation matrix, denoted $T$.
There is no unified notation to indicate in which coordinate system a point resides, nor is there to indicate from which to which coordinate systems a $T$ will transform.
% There is no unified notation for $T$ to indicate its source and target coordinate systems.
In order to dispel any confusion, the notation used in this work is explained below.
As is typical, right subscripts are used to indicate information about the owning quantity.
To indicate the coordinate system in which point or vector exists, the left superscript is used.
\begin{equation*}
	\prescript{0}{}p_i
\end{equation*}
represents the $i$-th point in coordinate system 0 (shorthand for the global coordinate system).
For transformation matrices the right subscript is designates the \textit{source} coordinate system, and the left superscript the \textit{target} coordinate system.
This leaves the right superscript position open for a transpose or inverse symbol.
\begin{equation*}
	\prescript{0}{}T_{A}
\end{equation*}
gives a transformation matrix from the \textit{A} coordinate system to the global coordinate system.
This schema provides an inherent check that multiplication between two quantities is valid according to their frames of reference.
\begin{equation*}
	\prescript{0}{}p_i = \prescript{0}{}T_{A} \prescript{A}{}p_i
\end{equation*}
shows the transformation of point $p_i$ in coordinate system $A$ to the global coordinate system.
In this last example the coordinate system of $p_i$ and the source coordinate system of $T$ appear adjacent and near vertical of one another, confirming that this is a valid transformation.

\section{Geometric Curvature}
The first step of \textit{3D Segmentation} is to segment the object's mesh using Watershed segmentation.
Watershed segmentation requires that a ``height'' function be applied to each vertex of the given mesh.
See Section~\ref{sec:ws_seg} for greater detail on Watershed segmentation.
In order to segment the mesh along regions of high curvature, the vertices' height must be a function of their curvature.
The various forms and types of curvature examined throughout this project are presented in the following sections.

Curvature describes the rate of change of a curve or surface's tangent.
Figure~\ref{fig:tangent_on_curve} shows the curvature $\kappa$ as the change of the tangent at the marked point as it moves along the curve.
Alternatively, the curvature of a point on a surface can be defined via the osculating circle at that point, which is the circle that best approximates the surface at said point.
Following that definition, the curvature is the inverse radius of the osculating circle.
An example of an osculating circle and its relation to curvature can be found in Figure~\ref{fig:principal_k}.

\begin{figure}[ht]
	\centering
\begin{tikzpicture}[scale=2.0]
	% based on image in slide 22/23 in DDG-Curves.pdf
	% \draw[step=10mm,gray!50,very thin] (-0.2,-0.2) grid (5.2,3.2);
	\clip (-0.5,1.1) rectangle (3,-0.1);
	\draw[] (-0.1,-0.1) .. controls (2.0,2.0) and (2.5,0) .. (1.5,0)
		% node [sloped,pos=0.1,minimum size=10mm,anchor=south west] (pt1) {}
		% node [sloped,pos=0.15,minimum size=9mm,anchor=south west] (pt1d) {}
		node [sloped,pos=0.38,minimum size=15mm,anchor=south west] (pt2) {}
		node [sloped,pos=0.43,minimum size=12mm,anchor=south west] (pt2d) {};
	% Point 1
	% \fill (pt1.south west) circle[radius=2pt,blue];
	% \path (pt1.south west) edge[-Stealth] node[above] {$\vec{t}$} (pt1.south east);
	% Point 1 delta and curvature
	% \path (pt1d.south west) edge[-Stealth] node[] {} (pt1d.south east);
	% \draw[dashed] (pt1.south east) -- (pt1d.south east) node[pos=0.5] {$\kappa$};
	% Point 2
	\fill (pt2.south west) circle[radius=1.5pt,blue];
	\path (pt2.south west) edge[-Stealth] node[above] {$\vec{t}$} (pt2.south east);
	% Point 2 delta and curvature
	\path (pt2d.south west) edge[-Stealth] node[] {} (pt2d.south east);
	\draw[red] (pt2.south east) -- (pt2d.south east) node[pos=0.5,anchor=west] {$\kappa$};
\end{tikzpicture}
	\caption{%
Normal vector of a point on a curve.
Inspired by~\cite{DDGAppIntro_12_smooth_curves}.}
	\label{fig:tangent_on_curve}
\end{figure}

Initially, the root mean square (RMS) curvature calculated from discrete Gaussian and mean curvature was used as the mesh height.
This was chosen following Pulla et al.'s work showing benefit of RMS curvature for Watershed segmentation~\cite{Imp_k_estimation_for_WS}.
This was eventually replaced with Taubin's method of calculating the principal curvatures (described below) directly via curvature tensor estimation~\cite{TaubinTensor}.
This has the benefit of direct calculation, but has been shown to be susceptible to noise in the mesh~\cite{Comp_k_notes}.
% Thiesel et al.\ calculate the curvature tensor per triangular face in the mesh, based on the face's corner normals~\cite{Norm_based_k_tensor_est}.
This too was later retired, and replaced with Rusinkiewicz's approach of calculating the curvature tensor for each mesh face via the differences between corner normals~\cite{SRTensor}, then computing the per vertex curvature tensor as a weighted sum over the curvature tensors of the vertex's adjoining triangles via coordinate system transformations.
% Gatzke and Grimm examine a variety of other curvature estimation methods~\cite{EstCurvOnTriMesh}, most of which were not considered for this work.

\subsection{Principal Curvatures}
All combined measures of curvature are based, either in theory or in actuality, on the principal curvatures.
The principal curvatures at a given point on a surface are simply the maximum $\kappa_1$ and minimum $\kappa_2$ curvature~\cite{DDGAppIntro_17_smooth_k}.
Note that in this work the first principal curvature is the maximum and the second the minimum, whereas other sources may use the reverse schema.
The direction of a principal curvature is known as a principal direction, and is typically denoted $X_i$.
The principal curvatures and directions at the top of a hill-like surface are illustrated in Figure~\ref{fig:principal_k}.

\begin{figure}[htb]
	\centering
	% \includegraphics[width=0.7\textwidth]{../resources/minimal_surface_curvature_planes-en.svg.png}
	\includegraphics[width=0.7\textwidth]{../resources/curvature/principal_curvatures.png}
	% \includesvg[width=0.9\textwidth]{../resources/Minimal_surface_curvature_planes-en.svg}
	\caption{
Principal curvatures are shown on an ellipsoidally round surface~\cite{Digital_geom_proc_w_disc_ext_calc}.
$X_1$ and $X_2$ show the principal directions.
The cutaway views show this as an osculating circle in each projection of the surface.
}
	\label{fig:principal_k}
\end{figure}

Determining principal curvatures for a continuous surface is unambiguous.
The principal curvatures of a surface $S$ at point $p$ are defined by the 2nd fundamental form $\textbf{II}(X,Y)$, where $X$ and $Y$ are orthonormal tangent vectors at $p$~\cite{DiffGeo_curves_surfaces, Basic_diff_geo_of_surfaces, DDGAppIntro_17_smooth_k}.
\iffalse
\begin{equation}
	\textbf{II}(U,V) = Ldu^2 + 2 M du dv + N dv^2,%\langle dN(X), df(Y)\rangle,
\end{equation}
where $U$ and $V$ are orthonormal vectors tangent to the point $(u,v)$ on some surface, $du$ and $dv$ are the changes in the $U$ and $V$ directions, respectively.
Additionally, $L$ is ...
and can be found as the eigenvalues of
\fi
The principal curvatures are the eigenvalues of
\begin{equation}\label{eq:2nd_fundamental_tensor}
	\begin{bmatrix}
		\textbf{II}(X_1, X_2) & \textbf{II}(X_1, X_2) \\
		\textbf{II}(X_2, X_1) & \textbf{II}(X_2, X_2)
	\end{bmatrix},
\end{equation}
where $X_1$ and $X_2$ are the aforementioned orthonormal vectors tangent to the surface at point $p$.
Similarly, the principal directions are the eigenvectors of Equation~\eqref{eq:2nd_fundamental_tensor}.

There are, however, various ways of approximating the principal curvatures on a discrete mesh~\cite{EstCurvOnTriMesh, DiscDiffGeoOpsTriMani}.
A basic approach is to calculate the mean and Gaussian curvatures, and derive the principal curvatures from them~\cite{DDGAppIntro_19_discrete_k_2, Gauss_mean_k_notes}.
Given the mean curvature $H$ and Gaussian curvature $K$ (see Sections \ref{sec:mean_k} and \ref{sec:gauss_k}), the principal curvatures can be calculated from
% Given the mean curvature $H$ and Gaussian curvature $K$ (see Sections \ref{sec:mean_k} and \ref{sec:gauss_k}), the principal curvatures can be calculated from Equations \ref{eq:mean_k} and \ref{eq:gauss_k}, or the discrete approximations from Equations \ref{eq:dihedral_angle} and \ref{eq:disc_gauss_k}.
\begin{align*}
	\kappa_1 &= H - \sqrt{H^2 - K}, \\
	\kappa_2 &= H + \sqrt{H^2 - K}.
\end{align*}
Although theoretically impossible, discretization errors can cause $H^2$ to be less than $K$, resulting in ``imaginary'' curvature.
This tends to occur on planar mesh regions, thus a minimum of 0 can be set for $H^2 - K$, because the root term would have been near 0 regardless, but it does highlight a flaw in this method of calculating prinicpal curvature.

\subsection{Mean Curvature}\label{sec:mean_k}
The discrete mean curvature was used early in development as part of the RMS curvature calculation.
Mean curvature is, as the name implies, the mean of the principal curvatures~\cite{DDGAppIntro_19_discrete_k_2}.
The mean curvature calculation is
\begin{equation*}%\label{eq:mean_k}
	H = \frac{\kappa_1 + \kappa_2}{2},
\end{equation*}
where $H$ is the mean curvature.
Mean curvature can be approximated on a discrete mesh via the sum of dihedral angle and edge length products:
\begin{equation}\label{eq:dihedral_angle}
	H_i := \frac{1}{2}\sum_{j \in E}l_{ij} \phi_{ij}
\end{equation}
where $H_i$ is the mean curvature at vertex $i$, $j$ is a vertex in the neighborhood of $i$, $l_{ij}$ is the length of the edge from $i$ to $j$, and $\phi_{ij}$ is the dihedral angle between the faces adjacent to edge $ij$.
See Figure~\ref{sfig:dihedral_angle} for a graphic depiction.

Alternatively, the mean curvature normal $\Delta f$ for the vertex $i$ can be calculated via the discrete Laplace-Beltrami operator~\cite{DDGAppIntro_18_discrete_k_1}
\begin{equation}\label{eq:lap_beltrami_op}
	(\Delta f)_i := \frac{1}{2}\sum_{j \in E}(\cot \alpha_{ij} + \cot \beta_{ij})(p_j - p_i),
\end{equation}
where each $j$ is a vertex in vertex $i$'s neighborhood, $p_i$ and $p_j$ are the 3D positions of vertices $i$ and $j$, and $\alpha_{ij}$ and $\beta_{ij}$ are the angles opposite the edge from $i$ to $j$.
The spatial relation of vertices $i$, $j$, and angles $\alpha_{ij}$ and $\beta_{ij}$ is illustrated in Figure~\ref{sfig:mesh_neighborhood}.
A vertex's neighborhood is the group of vertices directly adjacent it.
% Note that some sources will call this a ``1-ring'', meaning the ring of vertices 1 edge away from the center vertex.
The absolute mean curvature can be calculated as half of the magnitude of the mean curvature normal using
\begin{equation*}
	|H_i| = \frac{\|(\Delta f)_i \|}{2},
\end{equation*}
where $H_i$ is the mean curvature for vertex $i$ and $(\Delta f)_i$ is the mean curvature normal defined in Equation~\eqref{eq:lap_beltrami_op}.

\begin{figure}[htb]
	\centering
	\begin{subfigure}[t]{0.47\textwidth}
		\centering
\begin{tikzpicture}[
	hidden/.style={inner sep=0, outer sep=0}]
	\node [vertex,label=180:i] (node_i) at (-0.4,2.2) {};
	\node [vertex,label=215:j] (node_j) at (0.4,-2.2) {};
	\node [hidden] (node_l) at (-2.3,-0.8) {};
	\node [hidden] (node_r) at (1.7,0.5) {};
	\draw (node_i) -- (node_j) -- (node_r) -- (node_i) -- (node_l) -- (node_j);
	\draw [dashed,gray!60] (node_l) -- (node_r);
	% try to draw curved arrow
	\draw [-Stealth,thick] (20:4mm) arc [start angle=20, delta angle=-180, x radius=5mm, y radius=3mm];
		% node [pos=0.8, label={250:$\phi_{ij}$}] {};
	\node [] at (-0.3, -0.6) {$\phi_{ij}$};
	\draw [|-|, thick] ($(node_i.east)+(0.1,0.05)$) -- ($(node_j.east)+(0.1,0.05)$)
		node [pos=0.65,label={0:$l_{ij}$}] {};
\end{tikzpicture}
		\caption{Shown is the intersection of two triangular faces, with their shared edge and dihedral angle marked.}
		\label{sfig:dihedral_angle}
	\end{subfigure}
	\hfill
	\begin{subfigure}[t]{0.47\textwidth}
		\centering
\begin{tikzpicture}[scale=1.1]
	% based on image in slide 26/44 in DDG-DiscreteCurvatureI.pdf
	\newdimen\hexR
	\hexR=2cm
	\newlength\arcR
	\arcR=4mm
	\node[vertex,label=south west:i] (0, 0) {};
	\draw [thick] (330: \hexR) \foreach \x in {30,90,...,330} { -- (\x:\hexR) };
	\foreach \x in {30,90,...,330} {
		\draw (0, 0) -- (\x:\hexR) node[vertex]{};
	}
	% Vertex j
	\node [label=270:j] at (270:\hexR) {};
	\node [label=330:k] at (330:\hexR) {};
	% Inner main angle
	\draw (270:\arcR) arc [radius=\arcR, start angle=270, end angle=330]
		node [pos=0.5, label={[xshift=-3.5mm, yshift=1mm]300:$\Theta{ijk}$}] {};
	% Inner angles
	\draw (210:\hexR) +(330:\arcR) arc [radius=\arcR, start angle=-30, end angle=30]
		node [pos=0.5, label={[xshift=-2.2mm]0:$\alpha_{ij}$}] {};
	\draw (330:\hexR) +(150:\arcR) arc [radius=\arcR, start angle=150, end angle=210]
		node [pos=0.5, label={[xshift=2.5mm]180:$\beta_{ij}$}] {};
\end{tikzpicture}
		\caption{Shown is vertex i's neighborhood, with noteworthy angles marked.}
		\label{sfig:mesh_neighborhood}
	\end{subfigure}
\caption{
Subfigure (a) depicts angle $\phi_{ij}$ and edge length $l_{ij}$ from Equation~\eqref{eq:dihedral_angle}, where they are used to calculate mean curvature.
Subfigure (b) portrays the opposite angles $\alpha_{ij}$ and $\beta_{ij}$ across from edge $ij$.
The opposite angles, as well as the positions of vertices $i$ and $j$, are used in Equation~\eqref{eq:lap_beltrami_op} to calculate the mean curvature normal.
Subfigure (b) also shows $\Theta_{ijk}$, the angle between adjacent edges used in Equation~\eqref{eq:disc_gauss_k} to calculate the angle deficit of vertex $i$.
Both illustrations were based on similar ones in~\cite{DDGAppIntro_19_discrete_k_2}.
}
\end{figure}

\subsection{Gaussian Curvature}\label{sec:gauss_k}
Similar to mean curvature, discrete Gaussian curvature was used as part of the RMS curvature calculation.
Gaussian curvature is the product of the principal curvatures and can be calculated using
\begin{equation*}%\label{eq:gauss_k}
	K = \kappa_1 \cdot \kappa_2,
\end{equation*}
where $K$ is the Gaussian curvature~\cite{TheoremaEgregium}.
The discrete Gaussian curvature at a vertex is typically approximated as the ``angle defect'' divided by the vertex's area using
\begin{equation}\label{eq:disc_gauss_k}
	K = \frac{2\pi - \sum \Theta_j}{A_i}
\end{equation}
where $\Theta_j$ is the angle between adjacent edges from the central vertex ($\Theta_{ijk}$ in Figure~\ref{sfig:mesh_neighborhood}), and $A_i$ is the vertex area.

\begin{figure}[htb]
	\centering
	% TODO: create a tikz picture to replace this image
	\includegraphics[width=0.9\textwidth]{../resources/curvature/gaussian_mean_k.png}
	\caption{This Comparison of Gaussian (a) and mean (b) curvature demonstrates how a surface that bends in only one direction will have 0 Gaussian curvature, whereas the mean curvature will vary with that direction~\cite{Imp_k_estimation_for_WS}.}
	\label{fig:mean_gauss_k} % NOTE: Label should be declared after caption
\end{figure}

Mean and Gaussian curvature are compared visually in Figure~\ref{fig:mean_gauss_k}.
Note that the Gaussian curvature is approximately 0 over the entire surface, because the surface curves in only 1 direction, thus the 2nd principal curvature is \textasciitilde0.
The colors shown in the mean curvature plot are effectively due entirely to the 1st principal curvature values.

\subsection{Root Mean Square Curvature}
Pulla et al.\ compared different types of curvature for the purpose of Watershed segmentation and found that although the absolute curvature produced the best results, the root mean square curvature produced similar results and was less computationally expensive~\cite{Imp_k_estimation_for_WS}.
The root mean square curvature is, as the name implies, the square root of the average of the principal curvatures squared:
\begin{equation*}
	\kappa_{rms} = \sqrt{\frac{\kappa_1^2 + \kappa_2^2}{2}}.
\end{equation*}
% They compared different measures of curvature for this purpose and found that it was approximately as good as absolute curvature:
% \begin{equation*}
% 	\kappa_{abs} = |\kappa_1| + |\kappa_2|,
% \end{equation*}
They found $\kappa_{rms}$ cheaper to compute than the absolute curvature because they computed it from the mean and Gaussian curvatures using
\begin{equation*}
	\kappa_{rms} = \sqrt{4H^2 - 2K},
\end{equation*}
where $H$ and $K$ are the mean and Gaussian curvatures explained above.
% RMS curvature was used during developement prior to implementing the derivative of curvature.

\subsection{Derivative of Curvature}
Through testing it was proposed that supplying the derivative of curvature instead of the curvature itself to watershed segmentation would better preserve the boundaries between semantic regions.
Rusinkiewicz proposes a method of taking the derivative of the curvature tensor~\cite{SRTensor}.
Because watershed segmentation expects a single value rather than a tensor, the magnitude of the derivative of the curvature tensor was approximated as the sum of squares of said tensor.

\section{Surface Regression}
The second high-level step within \textit{3D Segmentation} is \textit{Surface Classification}, which attempts to classify the mesh segments yielded by Watershed segmentation as belonging to one of the defined geometric primitives.
Fitting 3D points to the various primitives was a part of \textit{Surface Classification}, initially as a means of confirming initial shape primitive estimations, and later as means of direct shape primitive classification.
In both uses the regression results were confirmed by comparing the error value, obtained from the difference between the fitted primitive and the set of 3D points, to a pre-defined threshold.
The methods to perform and validate regression on the shape primitives implemented in this work are described in the following sections.

\subsection{Planar Regression}\label{sec:planar_regression}
A plane may be defined by a normal vector and a point on the plane.
Given a set of 3D points, principal component analysis (PCA) can be used to calculate the plane's normal vector as the PCA's least principal component.
The mean of the points is sufficient to produce the intersection point on the plane.
Calculating the error of such a regression is done by computing the distance from each point to the plane and calculating the root mean square of said distances according to
\begin{equation*}
	e_{RMSE} = \left(\frac{1}{N}\sum_{i}^{N}d_i^2 \right)^{\frac{1}{2}}.
\end{equation*}
% The distance from a point $p_i$ to a plane described by point $p_P$ and normalized normal vector $\vec{n}$ may be calculated by transforming the point $p_i$ to the plane's coordinate system, with the origin at $p_P$ and which $\vec{n} = \vec{e_z}$.
The distance from a point $p_i$ to a plane described by point $p_P$ and normalized normal vector $\vec{n}$ may be calculated by transforming the target point to the plane's coordinate system, and taking the transformed point's $z$ coordinate as the distance, as shown by
% The distance from a point $p_i$ to a plane may be calculated by transforming the target point to the plane's coordinate system, with the origin at $p_P$ and which $\vec{n} = \vec{e_z}$:
\begin{equation*}
	\prescript{P}{}p_i = \prescript{P}{}T_0 \prescript{0}{}p_i
\end{equation*}
\begin{equation*}
	d_i = \vec{e_z} \cdot \prescript{P}{}p_i
\end{equation*}
where $\prescript{P}{}T_0$ is the homogeneous transformation matrix from global to planar coordinate systems.
Alternatively, if there is no other reason to compute the transformation matrix for the plane's local coordinate system, it is simpler to calculate the point-plane distance as the dot product of the plane normal with the vector $p_i - p_P$ as shown by
\begin{equation*}
	d_i = \langle p_i - p_P, \vec{n}\rangle.
\end{equation*}

\subsection{Cylindric Regression}
Cylinders in this context can be understood as an extruded conic section.
While there are methods of fitting a cylinder to a mere collection of 3D points~\cite{PCL_cyl_regression}, the process employed here was simplified by exploiting the mesh's normals.
For a cylindric mesh of constant radius with vertex normals, the normals will all be perpendicular to the main axis.
Thus, the least principal component from a PCA performed on the mesh's normals will yield a vector along the main axis.
From here a coordinate system may be created such that the z axis is the cylinder's main axis.
The origin of the cylindric coordinate system is for the cylinder, and the ensuing conic regression, inconsequential, but is typically set initially to the mean point position of the mesh.
Next, the mesh vertices are projected onto the XY plane of the cylinder's coordinate system, so that the specific type of conic may be deduced.
The general equation of a conic is:
\begin{equation}\label{eq:gen_conic}
	Ax^2 + Bxy + Cy^2 + Dx + Ey + F = 0.
\end{equation}
% Treating this as the error function to be minimized via least squares
Fitting a given set of 2D points to this equation can be achieved via total least squares.
Rosin shows that it is useful to set $F=1$~\cite{Ellipse_least_squares}, both to normalize the equation, and to avoid the trivial solution
\begin{equation*}
	A = B = C = D = E = F = 0.
\end{equation*}
Once the coefficients have been obtained, the specific type of conic represented by the points may be discerned by the determinant
\begin{equation*}
	B^2 - 4AC.
\end{equation*}
The specific method and equations to check the validity of the conic regression depend on the type of conic.

\subsubsection{Elliptic Regression}\label{sec:elliptic_reg}
For $B^2 - 4AC < 0$ the set of points represent an ellipse.
Computing the distance from a given point to an ellipse or other conic shape is non-trivial and, at the time of writing, no exact algebraic solution was found.
In place of an exact distance function, the sum of squared residuals was used throughout most of the project's timeline.
For example, a point on a conic section should satisfy Equation~\eqref{eq:gen_conic}, but for a point \textit{almost} on a conic section, the result will be non-zero:
\begin{equation}
	R(x,y) = Ax^2 + Bxy + Cy^2 + Dx + Ey + F \neq 0,
\end{equation}
where $R(x,y)$ is known as the residual.
Using each point's residual as its error value, the error function may be written as
\begin{equation}\label{eq:sq_residual}
	e = \sum_i \left(Ax_i^2 + Bx_i y_i + Cy_i^2 + Dx_i + Ey_i + F\right)^2.
\end{equation}
While easy to calculate, this value is unreliable as an indicator of whether or not the given surface is actually a conic surface.
During testing cases were encountered in which obviously non-cylindric surfaces yielded extremely low error values, resulting in false positives.
This realization fueled the need for a proper point-to-ellipse distance function.

Eberly~\cite{GeoTools_pt_to_ellipse} shows that for an origin-centered axis-aligned ellipse described by major and minor radii $a$ and $b$, and a target point $(x_p, y_p)$, the distance $t$ from the target point to the ellipse must satisfy the equation
\begin{equation}\label{eq:ellipse_dist}
	F(t) = \left(\frac{a x_p}{t + a^2}\right)^2 + \left(\frac{b y_p}{t + b^2}\right)^2 - 1 = 0.
\end{equation}
While Equation~\eqref{eq:ellipse_dist} has no direct analytic solution, Eberly shows 3 methods for computing it numerically.
He discusses the merits of each and concludes that the bisection method is best due to its robustness.
% also compares 3 methods of solving Equation~\eqref{eq:ellipse_dist} for $t$, concluding that the bisection method is the most robust.
\vspace{1cm}
% TODO: figure out a more elegant way to force the TSP section after the RDP pseudocode...

\subsubsection{Parabolic and Hyperbolic Regression}
% TODO: return to this, improve it.?
The functionality to perform parabolic or hyperbolic regression was omitted from this work due to time constraints.

\section{UV Mapping}
\begin{figure}[htb]
	\centering
	\includegraphics[width=0.65\textwidth]{../resources/cube_UV_unwrapping.png}
	\caption{Visual depiction of a cube's UV map being unwrapped~\cite{UV_map_cube_img}.}
	\label{fig:uv_map_cube}
	% NOTE: image requested by HK
\end{figure}
UV Mapping is the process of projecting a surface from 3D to 2D, effectively ``unwrapping'' the 3D surface.
UV maps are used extensively in 3D modeling and computer graphics to apply textures to models.
Figure~\ref{fig:uv_map_cube} shows the UV map of a cube being unwrapped.
Within the context of this work the primary purpose of UV maps are to handle the transformation of points from global to surface coordinates.
The plan was to have a UV map for each shape primitive to use during the high-level step \textit{Geometry Simplification}.
% The functions a "UV-Map" in the code accomplishes:
	% Eigen::Vector2d xyz_to_uv(const Eigen::Vector3d& vec3) const;
	% Eigen::Vector3d uv_to_xyz(const Eigen::Vector2d& vec2) const;
	% Eigen::Vector3d pt_dir_intersection(
	% 	const Eigen::Vector3d& pt, const Eigen::Vector3d& dir) const;
	% Eigen::Vector3d surface_normal_at_pt(const Eigen::Vector3d& pt) const;
	% Eigen::Vector3d theta_to_RxRyRz(double theta) const;

The UV map for planar surfaces is trivial, as the 3D surface is already ``unwrapped''.

\subsection{Cylindric Surfaces}
For the purposes of this project, cylindric surfaces are understood to be extruded ellipses.
The process of unwrapping a cylinder is first described for a circular cylinder, and then expanded to the elliptic case.

% \vspace{2cm}
\subsubsection{Circular Cylinders}
The transformation matrix $\prescript{c}{}T_0$ is the core of a circular cylinder's UV map.
The z axis of $\prescript{c}{}T_0$ aligns with the cylinder's axis, and its origin is positioned at one end of the cylinder.
An example thereof can be seen in Figure~\ref{sfig:gl_ccyl_transformation} below.
The transformation procedure is described visually in Figure~\ref{fig:gl_ccyl_transform_steps}, and textually below.
\begin{figure}[htb]
	\centering
	\begin{subfigure}[b]{0.3\textwidth}
		\centering
\begin{tikzpicture}[scale=1.0]
	% CSYS
	\draw[->] (0,0) -- +(210:1.5) node[left] {X};
	\draw[->] (0,0) -- +(-30:1.5) node[right] {Y};
	\draw[->] (0,0) -- (0,1.5) node[above] {Z};
	% Cylinder
	\node[cylinder,shape border rotate=90,
		aspect=2.5, minimum height=20mm, minimum width=20mm,
		draw=blue!60] at (0,0.0) {};
	% Point
	\fill[purple] (1,0.5) circle[radius=3pt];
\end{tikzpicture}
		\caption{%
Step 1: Transform point to cylinder coordinates.
\begin{equation*}
	\prescript{c}{}p = \prescript{c}{}T_0 \prescript{0}{}p
\end{equation*}
}
		\label{sfig:gl_ccyl_transformation}
	\end{subfigure}
	\hfill
	\begin{subfigure}[b]{0.3\textwidth}
		\centering
\begin{tikzpicture}[scale=1.0]
	% CSYS
	\draw[thin,->] (0,0) -- (1.4,0) node[right] {X};
	\draw[thin,->] (0,0) -- (0,1.4) node[above] {Y};
	% Cylinder -> Circle
	\draw (0,0) circle[radius=10mm,blue];
	% Point
	\node[circle,minimum size=6pt,fill=purple,inner sep=0, outer sep=0,label=135:$p$] (p1) at (135:10mm) {};
	% arrow and angle
	\draw (0,0) -- (p1);
	\draw[-stealth,black] (8mm,0) arc[start angle=0, end angle=135, radius=8mm]
		node[pos=0.37,below left] {$\phi$};
\end{tikzpicture}
		\caption{%
Step 2: Calculate $\phi$ using the point's x and y.
\begin{equation*}
	\phi = \arctan \frac{y}{x}
\end{equation*}
}
		% \label{fig:tangent_on_curve}
	\end{subfigure}
	\hfill
	\begin{subfigure}[b]{0.3\textwidth}
		\centering
\begin{tikzpicture}[scale=1.0]
	% CSYS
	\draw[thin,->] (0,0) -- (3.5,0) node[right] {U};
	\draw[thin,->] (0,0) -- (0,2.1) node[above] {V};
	% Grid to represent unwrapped surface
	\draw[xstep=4mm,ystep=9mm,very thin] (0,0) grid (3.2,1.8);
	% Point
	\fill[purple] (1.178,1.35) circle[radius=3pt];
\end{tikzpicture}
		\caption{%
Step 3: Calculate UV coordinates.
% Point $\prescript{c}{}p$ shown on the unwrapped cylinder.
\begin{equation*}
	p(u,v) = (L \frac{\phi}{2\pi}, \prescript{c}{}p\text{.z})
\end{equation*}
}
		% \label{fig:tangent_on_curve}
	\end{subfigure}
	\caption{Graphic depiction of a point transformed from global to cylinder surface coordinates.}
	\label{fig:gl_ccyl_transform_steps}
\end{figure}

First the point is transformed to the cylinder's coordinate system in $R^3$.
From here the $x$ and $y$ coordinates of the point are used to calculate the angle $\phi$ measured from the $x$ axis.
The point's $u$ coordinate is calculated as the ratio of $\phi$ to its perimeter using
\begin{equation}\label{eq:cyl_u_coord}
	u = L \frac{\phi}{2\pi},
\end{equation}
where $L$ is the cylinder's perimeter: $2\pi r$.
% $\phi$ is used to calculate the arc length position of $\prescript{c}{}p$ along the cylinder's perimeter, which becomes its $u$ position in surface coordinates.
The point's $z$ coordinate in the cylinder's coordinate system becomes its $v$ coordinate in surface coordinates.

To transform a point in surface coordinates to $R^3$, the steps above must be reversed.
Solving Equation~\eqref{eq:cyl_u_coord} for $\phi$, the point's $u$ coordinate is converted back to the angle $\phi$:
\begin{equation*}
	\phi = 2\pi \frac{u}{L}.
\end{equation*}
Next the point's position in cylinder coordinates are calculated as
\begin{equation*}
	\prescript{c}{}p(x,y,z) = (r \cos \phi, r \sin \phi, v)
\end{equation*}
where $r$ is the circle's radius.
Finally, the point may be transformed from the cylindric to the global frame of reference using the transformation matrix $\prescript{0}{}T_c$.

\subsubsection{Elliptic Cylinders}
An ellipse is merely a circle that has been stretched.
Because of this, the procedure to transform a point from $R^3$ to the surface coordinates of an elliptic cylinder is the same as that of the circular cylinder, save for a different equation.
Here, the elliptic UV map contains transformation matrix $\prescript{e}{}T_0$, with its z axis aligned with the cylinder's main axis, x axis aligned with the ellipse's major axis, and origin centered on the ellipse.
Recall that the general equation of an ellipse is
\begin{equation*}
	\frac{x^2}{a^2} + \frac{y^2}{b^2} = 1,
\end{equation*}
where $a$ is the radius along the major axis, and $b$ is the radius along the minor axis.
The transformation procedure is described visually in Figure~\ref{fig:gl_ecyl_transform_steps}, and textually further below.

\begin{figure}[htb]
	\centering
	\begin{subfigure}[b]{0.3\textwidth}
		\centering
\begin{tikzpicture}[scale=1.0]
	% \draw[step=10mm,gray!50,very thin] (-2.2,-0.2) grid (2.2,3.2);
	% CSYS
	\draw[->] (0,0) -- +(210:1.5) node[left] {X};
	\draw[->] (0,0) -- +(-30:1.5) node[right] {Y};
	\draw[->] (0,0) -- (0,1.7) node[above] {Z};
	% Ellipse(s)
	% \draw[gray!50] (0,0.0) ellipse[x radius=12mm, y radius=6mm,rotate=30];
	\draw[] (24:11.9mm) -- +(0,0.7);
	\draw[] (203:11.8mm) -- +(0,0.7);
	\draw (26:12.0mm) {[rotate=30] arc
		[start angle=-13, delta angle=-180, x radius=12mm, y radius=6mm,rotate=30]};
	\draw[] (0,0.7) ellipse[x radius=12mm, y radius=6mm,rotate=30];
	% Point
	\fill[purple] (1.1,0.7) circle[radius=3pt];
\end{tikzpicture}
		\caption{%
Step 1: Transform point to cylinder coordinates.
\begin{equation*}
	\prescript{e}{}p = \prescript{e}{}T_0 \prescript{0}{}p
\end{equation*}
}
		% \label{sfig:gl_ecyl_transformation}
	\end{subfigure}
	\hfill
	\begin{subfigure}[b]{0.3\textwidth}
		\centering
\begin{tikzpicture}[scale=1.0]
	% \draw[step=10mm,gray!50,very thin] (-2.2,-0.2) grid (2.2,3.2);
	% CSYS
	\draw[thin,->] (0,0) -- (1.5,0) node[right] {X};
	\draw[thin,->] (0,0) -- (0,1.5) node[above] {Y};
	% Cylinder -> Circle
	\draw[] (0,0) ellipse[x radius=12mm, y radius=6mm];
	% Point
	\node[circle,minimum size=6pt,fill=purple,inner sep=0, outer sep=0,label=135:$p$] at (-0.96,0.36) {};
	% arrow and angle
	% \draw (0,0) -- (p1);
	% \draw (8mm,0) arc [->,start angle=0, end angle=135, radius=8mm];
\end{tikzpicture}
		\caption{%
Step 2: Calculate $\phi$ using the point's x and y.
\begin{equation*}
	\phi = \arctan \frac{y/b}{x/a}
\end{equation*}
}
		\label{sfig:ellipse_xy_coords}
	\end{subfigure}
	\hfill
	\begin{subfigure}[b]{0.3\textwidth}
		\centering
\begin{tikzpicture}[scale=1.0]
	% CSYS
	\draw[thin,->] (0,0) -- (3.6,0) node[right] {U};
	\draw[thin,->] (0,0) -- (0,2.0) node[above] {V};
	% Grid to represent unwrapped surface
	\draw[xstep=3.4mm,ystep=8mm,very thin] (0,0) grid (3.4,1.6);
	% Point
	\fill[purple] (1.36,0.8) circle[radius=3pt];
\end{tikzpicture}
		\caption{%
Step 3: Calculate UV coordinates.
% Point $\prescript{c}{}p$ shown on the unwrapped cylinder.
\begin{equation*}
	p(u,v) = (L \frac{\phi}{2\pi}, \prescript{e}{}p\text{.z})
\end{equation*}
}
		% \label{fig:tangent_on_curve}
	\end{subfigure}
	\caption{Graphic depiction of a point transformed from global to cylinder surface coordinates.}
	\label{fig:gl_ecyl_transform_steps}
\end{figure}

The first step remains unchanged from the circular case: Transform the point to the cylinder's coordinate system.
The objective of the second step remains calculating the arc angle $\phi$, but here the coordinates must be scaled by their corresponding radii to account for the ellipse's eccentricity.
Note that for an ellipse $\phi$ is not equivalent to the angle to the point measured from the local $x$ axis, as that would be $\arctan y/x$.
Hence why there is no line or marker indicating $\phi$ in Figure~\ref{sfig:ellipse_xy_coords}.
The point's position in surface coordinates is given by $u$ from Equation~\eqref{eq:cyl_u_coord} and the $\prescript{e}{}p$'s z coordinate.

Performing the reverse transformation is equivalent to that of the circular cylinder, with the exception that the individual radii must be used instead of the singular radius $r$:
\begin{equation*}
	\prescript{c}{}p(x,y,z) = (a \cos \phi, b \sin \phi, v).
\end{equation*}
From here, the point is transformed from cylindric to global coordinates using $\prescript{}{}T_e$.

\iffalse
To obtain a vector perpendicular to a point on the ellipse, the derivative of said point is rotated 90 degrees.
The point, as a function of the angle $\theta$:
\begin{equation}
	(x,y) = (a\cos\theta, b\sin\theta)
\end{equation}
The derivative thereof:
\begin{equation}
	\frac{d}{d\theta}(x,y) = (-a\sin\theta, b\cos\theta)
\end{equation}
Rotated -90 degrees:
\begin{equation}
	\text{Rot}_{90}\frac{d}{d\theta}(x,y) = (b\cos\theta, a\sin\theta)
\end{equation}

\subsubsection{Parabolic and Hyperbolic Surfaces}
Parabolas are much simpler than and ellipses and hyperbolas, as can be seen by the following equations.
The parametric general equation of a parabola is:
\begin{equation}
	-\sin\theta x + \cos\theta y = \frac{1}{4f}(\cos\theta x + \sin\theta y - h)^2 + k
\end{equation}
To solve for the conic parameters of a parabola the general equation's quadratic term is expanded and like terms gathered:
\begin{multline*}
	-\sin\theta x + \cos\theta y = \frac{1}{4f}(\cos^2\theta x^2 + \cos\theta\sin\theta xy \\
	- h \cos\theta x + \cos\theta\sin\theta xy + \sin^2\theta y^2 - h\sin\theta y - h \cos\theta x - h \sin\theta y + h^2) + k
\end{multline*}
\begin{multline*}
	\frac{\cos^2\theta}{4f} x^2 + \frac{2\cos\theta\sin\theta}{4f} xy + \frac{\sin^2\theta}{4f} y^2 \\
	+ \left(\sin\theta - \frac{2h \cos\theta}{4f}\right)x + \left(\cos\theta - \frac{2h \sin\theta}{4f}\right)y + \frac{h^2}{4f} + k = 0
\end{multline*}
\begin{align}
	A &= \frac{\cos^2\theta}{4f} \\
	B &= \frac{2\cos\theta\sin\theta}{4f} \\
	C &= \frac{\sin^2\theta}{4f} \\
	D &= \sin\theta - \frac{2h \cos\theta}{4f} \\
	E &= \cos\theta - \frac{2h \sin\theta}{4f} \\
	F &= \frac{h^2}{4f} + k
\end{align}

\subsubsection{Hyperbolic Surface}
\begin{equation}
\begin{split}
	\frac{(\cos\theta(x-h) + \sin\theta(y-k))^2}{a^2} - \frac{(\cos\theta(y-k) + \sin\theta(x-h))^2}{b^2} &= 1 \\
	\frac{(x_c + y_s)^2}{a^2} - \frac{(y_c + x_s)^2}{b^2} &= 1 \\
	b^2(x_c^2 + 2x_c y_s + y_s^2) - a^2(y_c^2 + 2 x_s y_c + x_s^2) &= a^2 b^2 \\
	b^2 x_c^2 + 2 b^2 x_c y_s + b^2 y_s^2 - a^2 y_c^2 - 2 a^2 x_s y_c - a^2 x_s^2 - a^2 b^2 &= 0 \\
	b^2 x_c^2 - a^2 x_s^2 + 2 b^2 x_c y_s - 2 a^2 x_s y_c + b^2 y_s^2 - a^2 y_c^2 - a^2 b^2 &= 0 \\
\end{split}
\end{equation}
\begin{multline*}
	b^2 (\cos\theta(x-h))^2 - a^2 (\sin\theta(x-h))^2 \\
	+ 2 b^2 \cos\theta(x-h) \sin\theta(y-k) - 2 a^2 \sin\theta(x-h) \cos\theta(y-k) \\
	+ b^2 (\sin\theta(y-k))^2 - a^2 (\cos\theta(y-k))^2 - a^2 b^2 = 0
\end{multline*}
\begin{multline*}
	(b^2 \cos^2\theta - a^2 \sin^2\theta)(x-h)^2 \\
	+ 2 \cos\theta\sin\theta(b^2 - a^2)(x-h)(y-k) \\
	+ (b^2 \sin^2\theta - a^2 \cos^2\theta)(y-k)^2 - a^2 b^2 = 0
\end{multline*}
\begin{equation*}
	\begin{split}
		c_1(x-h)^2 + c_2(x-h)(y-k) + c_3(y-k)^2 - a^2 b^2 &= 0 \\
		c_1(x^2-2hx + h^2) + c_2(xy-hy-kx+hk) + c_3(y^2-2ky+k^2) - a^2 b^2 &= 0 \\
		c_1 x^2 + c_2 xy + c_3 y^2 + (-2h c_1 -k c_2) x + (-h c_2 -2k c_3)y + c_1 h^2 + c_2 hk + c_3 k^2 - a^2 b^2 &= 0 \\
	\end{split}
\end{equation*}
\begin{align*}
	A &= c_1 = b^2 \cos^2\theta - a^2 \sin^2\theta \\
	B &= c_2 = 2 \cos\theta\sin\theta(b^2 - a^2) \\
	C &= c_3 = b^2 \sin^2\theta - a^2 \cos^2\theta \\
	D &= (-2h c_1 -k c_2) \\
	E &= (-h c_2 -2k c_3) \\
	F &= c_1 h^2 + c_2 hk + c_3 k^2 - a^2 b^2
\end{align*}
Now to solve for $a$, $b$, and $\theta$:
Solving $A$ for $b^2$:
\begin{equation}
	\begin{split}
		A &= b^2 \cos^2\theta - a^2 \sin^2\theta \\
		b^2 &= \frac{A}{\cos^2\theta} + a^2\tan^2\theta \\
	\end{split}
\end{equation}
Setting 3.6 into $C$ and solving for $a^2$:
\begin{equation}
	\begin{split}
		C &= b^2 \sin^2\theta - a^2 \cos^2\theta \\
		a^2 &= -\frac{C}{\cos^2\theta} + b^2\tan^2\theta \\
		a^2 &= -\frac{C}{\cos^2\theta} + \left(\frac{A}{\cos^2\theta} + a^2\tan^2\theta\right)\tan^2\theta \\
		\cos^4\theta a^2 &= -C\cos^2\theta + A\cos^2\theta + \sin^4\theta a^2 \\
		(\cos^4\theta - \sin^4\theta) a^2  &= \cos^2\theta(A-C) \\
		a^2  &= \frac{\cos^2\theta(A-C)}{(\cos^4\theta - \sin^4\theta)} \\
	\end{split}
\end{equation}
Setting 3.6 into $B$:
\begin{equation}
	\begin{split}
		B &= 2 \cos\theta\sin\theta(\frac{A}{\cos^2\theta} + a^2\tan^2\theta - a^2) \\
		\cos^2\theta B &= 2 \cos\theta\sin\theta(A + (\sin^2\theta - \cos^2\theta) a^2) \\
	\end{split}
\end{equation}
Setting 3.7 into 3.8:
\begin{equation}
	\begin{split}
		\cos^2\theta B &= 2 \cos\theta\sin\theta(A + (\sin^2\theta - \cos^2\theta) \frac{\cos^2\theta(A-C)}{(\cos^4\theta - \sin^4\theta)}) \\
	\end{split}
\end{equation}
\fi

\section{Ramer-Douglas-Peucker Algorithm}\label{sec:RDP}
The Ramer-Douglas-Peucker (RDP) algorithm is a method to iteratively simplify a line defined by a set of points~\cite{RDP_line_reduction_DP, RDP_line_reduction_R}.
The only arguments to RDP are a list of points and a tolerance width, within which points will be considered unnecessary and discarded.
% The algorithm looks at a list of points, draws an imaginary line from the first to the last, finds the point with the farthest normal distance to the comparison line, and if the point's distance to the comparison line is greater than a pre-defined tolerance width, splits the lin
The algorithm iteratively finds the point farthest from the straight line from point 0 to n-1, and if said distance is greater than the tolerance width, the line is split at the farthest point and the line sections before and after the split are reconsidered individually.
The procedure is described in greater detail in Algorithm~\ref{alg:RDP}.

\begin{algorithm}[htb]
\caption{Ramer-Douglas-Peucker}\label{alg:RDP}
\begin{algorithmic}[1]
\Function{SimplifyLineRDP}{points $P$, real $w$}
	\State new list $P_{core}$\Comment{To store the important point indices}
	\State new RDPNode $n_0 \leftarrow$ RDPNode($P$.first, $P$.last)
	\State new list $N_{stack} \leftarrow n_0$
	\While{$N_{stack}$ not empty}
		\State new RDPNode $n \leftarrow N_{stack}$.last
		\State $N_{stack}$.popLast()
		\State new uint $i \leftarrow$ index of point farthest from $n$.line
		\State new real $d \leftarrow$ distance of $P$[$i$] from $n$.line
		\If{$d \le w$} \Comment{If farthest point is within tolerance}
			\State $P_{core}$.append($n$.srcPt) \Comment{Line section is complete}
			\State \textbf{continue}
		\EndIf
		\State new RDPNode $n_{left} \leftarrow$ RDPNode($n$.srcPt, $P$[$i$])
		\State new RDPNode $n_{right} \leftarrow$ RDPNode($P$[$i$], $n$.endPt)
		\State $N_{stack}$.append($n_{right}$)
		\State $N_{stack}$.append($n_{left}$)
	\EndWhile
	\State $P_{core}$.append($P$.last)
	\State \textbf{return} $P_{core}$
\EndFunction
\end{algorithmic}
\end{algorithm}

The RDP implementation used in this work (see Algorithm~\ref{alg:RDP}) uses a structure \textbf{RDPNode} that represents a section of the line.
It contains a start and end point, as well as a line spanning these two points.
A stack is used to manage the \textbf{RDPNode}s.
In each iteration of the \verb|while| loop starting on line 5, the item at the top of the stack is removed and stored in $n$.
The point farthest from this line section is found, and on line 10 the distance thereof is compared against the tolerance width.
If the perpendicular distance is within the tolerance width then the current line section is complete and the initial point in the section is added to $P_{core}$ (line 11).
If not, then the current line section is split at the farthest point, with the two new sections represented by RDPNodes $n_{left}$ and $n_{right}$.
These are then placed on the stack, $n_{right}$ first, so that $n_{left}$ is processed next, ensuring the list of important points $P_{core}$ is always sorted.
After the \textbf{RDPNode} stack is exhausted the last point in $P$ is appended to $P_{core}$ because only the start point in each node is added to $P_{core}$.
An alternative procedure that uses recursion may be found on Wikipedia.

\section{Traveling Salesman Problem}
The Traveling Salesman Problem (TSP) is a well known problem in combinatorial optimization.
The original problem was posed as the following:
A salesman wishes to visit every city in a given region once, to spend the least amount of time traveling between cities, and to end at his home city.
It was initially formulated in the 1800s by mathematicians William Rowan Hamilton and Thomas Kirkman~\cite{Graph_theory}.
The problem can be stripped down to simply the optimal round-trip traversal of a set of nodes in a graph, where the traversal represents the salesman's movement, and each node represents a city.
It can be shown~\cite{TSP_in_pursuit_of} that for $n$ nodes the number of possible route permutations is
\begin{equation*}
	n_{\text{routes}} = (n-1)!.
\end{equation*}
For the vast majority of applications symmetric node-to-node costs can be assumed, and the number of permutations is reduced to
% previous sentence previously ended with \textit{only}
% as a tiny bit of sass in this otherwise serious paper.
\begin{equation*}
	n_{\text{routes}} = \frac{(n-1)!}{2}.
\end{equation*}
The number of real world TSP applications is similarly large~\cite{TSP_theory_applications}, ranging from determining the drill order of holes in printed circuit board manufacturing~\cite{TSP_PCB_manufacturing}, to mail delivery and vehicle routing in general~\cite{TSP_mail_delivery}, to X-ray crystallography~\cite{TSP_xray_crystallography}.
The algorithm described in this work produces a set of convex surfaces from the target object (steps \textit{3D Segmentation} to \textit{2D Segmentation} in Figure~\ref{fig:bkgd_overview}).
Within each convex surface region a simple path is planned.
% NOTE the verb tense/case! "would" because it is not implemented
It is the traversal of these surface regions that would be formulated as a TSP with the distance between each region and the change in end-effector orientation serving as the travel costs.
Considering only the traversal of the surface regions, the start and end nodes need not be the same, and the situation can be modeled as a modified TSP.
The common way to modify a TSP to achieve different start and end nodes is to introduce a dummy node with 0-cost edges to the other nodes~\cite{TSP_dummy_node_mod}.
After insertion of the dummy node, the TSP can be solved like normal.
Once the TSP has been solved, the nodes adjacent to the dummy node are the start and end nodes.
If the robot's starting or home position is included as a node, the application can be handled as a classic TSP.



%! TeX root = thesis.tex
\chapter{Methodology}\label{methodology}
\IMRADlabel{methods}

\section{Overview}
The basic premise is to segment the given model into primitive surfaces.
Each surface is unwrapped to a 2D representation, which undergoes a 2D segmentation to produce convex regions.
Upon each convex region a local path is planned.
The order in which each region is traversed by the robot is determined via a modified Traveling Salesman Problem.
Here, the start and end points of the salesman's path need not be the same.

\section{3D Segmentation}
Segmentation in 3D consists of breaking the model into primitive surfaces,
which have (easily) solvable mappings from 3D to 2D.
This is done by applying Watershed segmentation to the mesh as a whole and classifying the resultant mesh sections as a certain primitive.
Watershed segmentation is imperfect, and sometimes yields mesh sections comprised of multiple primitive types.
In such cases the composite mesh section undergoes Watershed segmentation again, but with a lower merge threshold (see \ref{ws_seg}), so that it might be split into multiple mesh sections.

$\rightarrow$ Do i need to further describe Seg3D ?

\subsection{Watershed Segmentation}\label{ws_seg}
A watershed, according to the North American usage, is an area of land, in which all streams and rainfall drain to a common body of water\cite{USGS_Watersheds}, also commonly called catchment basins.
Somewhat confusingly, the rest of the English speaking world uses ``Watershed'' to refer to the high elevation regions that separate said catchment basins.
Watershed segmentation originally comes from image processing, where it is used for image segmentation\cite{ImageSegWS, DigitalImageProc}.
It works by applying a ``height function'' to the input image and forming image regions divided by ``high'' areas.
A common ``height function'' in image processing is the gradient of the image\cite{ImageSegWS}.
Mangan and Whitaker first applied this concept to 3D meshes, replacing pixels for the mesh's vertices and the image gradient for the mesh curvature\cite{Watershed}.

-> Give overview of how WS segmentation works.
\subsubsection{Basic Procedure}
Their algorithm consists of 6 steps:
\begin{enumerate}
	\item Apply the height function to each vertex
	\item Find and label each local minima
	\item Find flat areas and classify them as either a minimum or plateau
	\item Loop through the plateaus and allow each to descend to a labeled region
	\item Descend from all remaining vertices to labeled regions
	\item Merge regions whose watershed depth is below a given threshold
\end{enumerate}

0. Apply height function at each vertex
1. Create initial mesh regions from local minima / minima plateaus
2. ``Minima Expansion'' Expand outward from each mesh region up to a certain depth, absorbing any regions encountered.
3. ``Descent to Minima'' From each un-indexed (check their word for this) vertex, follow the path of steepest descent until an indexed region is encountered.
All vertices traversed along this descent are added to the encountered region.

-> my modifications
\subsubsection{Mini-Merge}
As noted by Mangan and Whitaker(sp?) Descent to Minima will segment the given mesh, but in all likelihood it will be overly segmented.
Through testing small high curvature regions were observed, that due to their high curvature did were not merged into any of the larger more useful regions.
To combat this, the ``Mini Merge'' step was developed to merge regions deemed too small to be worth keeping.

\subsubsection{Boundary Smoothing}
This step is more post-processing and ``cleaning'' of the mesh region boudnary than actual segmentation.
Due to randomness in the mesh there are situations where a vertex is connected to its region by a single edge.
Such vertices are named ``web1'' points, due to them having a single webbed connection to their region's perimeter.
In order to smooth the region boundary, an attempt is made to find an adjacent mesh region more suitable to possess each web1 point.
Because vertices with only 3 edges are exceedingly rare in well meshed models, it is sufficient to transfer ownership of the web1 point to the adjacent region with the highest number of connecing edges.
Thus, for example, A vertex assigned to region 4 through Minima Descent with edges connecting to regions 4, 10, 12, and 12, would be transferred to region 12.
No explicit tie breaking mechanism was deemed necessary, but lower numbered regions are likely given priority due to how the code was written.
TODO: Need to check what other cases could occur...

\subsection{Surface Classification}

\section{Geometry Simplification}
GeoSimp creates simplified geometric representations of the 3D mesh sections it is given

\subsection{Shared Edges}
see \verb|create_shared_edges()|

\subsection{Shared Corners}

\subsection{Simplified Surfaces}

\section{2D Segmentation}
This is Interior Edge Extension
The idea was that downstream components requried convex shapes to facilitate local path planning.
In hindsight, surfaces need not be completely convex, but merely \textit{mostly} convex.

\section{Surface Path Planning}
This is effectively Boustrophedon

\section{Modified Traveling Salesman Problem}
Normal TSP should have already been described in \ref{background}, so no need to rehash that.
Only need to talk about the modifications and how it would have been applied...

\section{Inverse Kinematics}
Talk about how the InvKin from the robot could be used, but a custom one would (likely) be necessary to incorporate the rotary table



%! TeX root = thesis.tex
\chapter{Evaluation}\label{sec:eval}
\IMRADlabel{results}
% Yes, i realize the chapter is "evaluation" and the IMRAD label is "results"
\section{Testing Setup}
% \subsection{Test Model Sources}
To test the algorithm developed, a collection of test meshes was sought.
The very early days of development relied on the default cube in Blender with varied mesh density.
Once Watershed segmentation yielded plausible results from the default cube, meshes from scanned objects were demanded.
To meet this request models were scoured from Fraunhofer IGD's own collection of scanned and reconstructed cultural artifacts.
Although useful to push the limits of the methods developed, cultural artifacts are not representative of the intended target objects: scrapped metal components.
All but one of the reconstructed meshes were too complex to be useful during testing.
To ensure the algorithm's generalizability additional test models were eventually sought, this time from CAD databases.
The vast majority of these came originally from Thingiverse, an online platform for users to post their 3D printable models~\cite{Thingiverse}.
The full list of test models and their sources can be found in Appendix~\ref{app:model_table}.

\subsection{Remeshing}
Because Watershed segmentation requires input models to have a mesh of vertices upon which the height function may be computed, and CAD models store only what is required to reconstruct their geometry, remeshing was required as a pre-processing step.
PyMeshLab's ``isotropic explicit remeshing'' was chosen for this purpose~\cite{PyMeshLab}.
Manually varying the mesh density by model region can reduce mesh file sizes and expedite computation time without compromising the quality of the path planned, but is still typically slower than applying a high-density mesh to the entire model and accepting slower algorithm completion times.
The primary input parameter of \textit{isotropic explicit remeshing} is the ``target length'', which sets the target length of the remeshed edges as a percentage of ``something''.
Unfortunately PyMeshLab's documentation is unclear as to what this value is a percentage of.
% I tried to find the answer, believe me.
The vast majority of test models were remeshed with a uniform ``target length'', and the remainder were remeshed using manual selection of regions.
Models that were remeshed with non-uniform target lengths have a range of values in the \textbf{Mesh} column in appendix \ref{app:model_table}.

Prior to testing each test object was examined and assigned a ``difficulty class'' from 1-4.
The difficulty classification was created during a period when the algorithm's conic regression and cylinder detection functionality were sub-optimal and required improvement.
Hence the higher difficulty class for objects with cylindric surfaces.
The example images in the following sections were created using Blender~\cite{Blender}.

\subsection{Difficulty Class 1}
These objects are comprised solely of planar surfaces and should pose little to no trouble to the path planner.
Figure \ref{fig:class_1_models} shows 3 such examples.

\begin{figure}[htb]
\centering
\begin{subfigure}{0.3\textwidth}
	\includegraphics[width=\textwidth]{../resources/models/fc028.png}
	\caption{Model fc028}
	% \label{sfig:first}
\end{subfigure}
\hfill
\begin{subfigure}{0.3\textwidth}
	\includegraphics[width=\textwidth]{../resources/models/1505020.png}
	\caption{Model 1505020}
	% \label{sfig:second}
\end{subfigure}
\hfill
\begin{subfigure}{0.3\textwidth}
	\includegraphics[width=\textwidth]{../resources/models/fc004.png}
	\caption{Model fc004}
	% \label{sfig:third}
\end{subfigure}
\caption{Test models of difficulty class 1}
\label{fig:class_1_models}
\end{figure}

\subsection{Difficulty Class 2}
Features that would cause an object to be pushed into class 2 include small or narrow surfaces, as well as cylindric surfaces.
There is a decent chance that small and thin surfaces will be absorbed into a neighboring large planar region, decreasing the chance that the combined region will be classified as planar.
Three examples of class 2 test models can be found in Figure \ref{fig:class_2_models}.

\begin{figure}[htb]
\centering
\begin{subfigure}{0.3\textwidth}
	\includegraphics[width=\textwidth]{../resources/models/55583.png}
	\caption{Model 55583}
	\label{sfig:55583}
\end{subfigure}
\hfill
\begin{subfigure}{0.3\textwidth}
	\includegraphics[width=\textwidth]{../resources/models/7120369.png}
	\caption{Model 7120369}
	\label{sfig:7120369}
\end{subfigure}
\hfill
\begin{subfigure}{0.3\textwidth}
	\includegraphics[width=\textwidth]{../resources/models/68501.png}
	\caption{Model 68501}
	% \label{sfig:third}
\end{subfigure}
\caption{Test models of difficulty class 2}
\label{fig:class_2_models}
\end{figure}

Of the models shown in Figure \ref{fig:class_2_models} model 55583 (Figure \ref{sfig:55583}) has the highest potential for failure, due to the narrow surfaces at the top of each gear tooth.
The smooth transitions from planar to cylindric in Figure \ref{sfig:7120369} will also be difficult for the path planner.

\subsection{Difficulty Class 3}
Objects of this class are expected to be difficult despite improvements to cylinder detection and regression, and handling of narrow planar surfaces.
The examples shown in Figure \ref{fig:class_3_models} showcase various difficult features.

\begin{figure}[htb]
\centering
\begin{subfigure}{0.3\textwidth}
	\includegraphics[width=\textwidth]{../resources/models/97733.png}
	\caption{Model 97733}
	% \label{sfig:first}
\end{subfigure}
\hfill
\begin{subfigure}{0.3\textwidth}
	\includegraphics[width=\textwidth]{../resources/models/42042.png}
	\caption{Model 42042}
	% \label{sfig:second}
\end{subfigure}
\hfill
\begin{subfigure}{0.3\textwidth}
	\includegraphics[width=\textwidth]{../resources/models/1148449.png}
	\caption{Model 1148449}
	% \label{sfig:third}
\end{subfigure}
\caption{Test models of difficulty class 3}
\label{fig:class_3_models}
\end{figure}

\subsection{Difficulty Class 4}
Difficulty class 4 models are beyond the scope of this work, but are included for comparison purposes.
Figure \ref{fig:class_4_models} shows a few such models.

\begin{figure}[htb]
\centering
\begin{subfigure}{0.3\textwidth}
	\includegraphics[width=\textwidth]{../resources/models/50704.png}
	\caption{Model 50704}
	\label{sfig:50704}
\end{subfigure}
\hfill
\begin{subfigure}{0.3\textwidth}
	\includegraphics[width=\textwidth]{../resources/models/229605.png}
	\caption{Model 229605}
	\label{sfig:229605}
\end{subfigure}
\hfill
\begin{subfigure}{0.3\textwidth}
	\includegraphics[width=\textwidth]{../resources/models/99265.png}
	\caption{Model 99265}
	\label{sfig:99265}
\end{subfigure}
\caption{Test models of difficulty class 4}
\label{fig:class_4_models}
\end{figure}

Although spherical surfaces, such as those adorning model 50704 (Figure \ref{sfig:50704}), were part of the planned set of detectable primitives, that functionality was not completed.
Thus, even if the path planner were able to perfectly handle the curved surfaces at the base of model 50704, the partial spheres would result in a failure.
Model 229605 (Figure \ref{sfig:229605} appears simple at first glance, but its lower half is that of a cone, and the detection thereof is not implemented.
Model 99265, shown in Figure \ref{sfig:99265}, exhibits swept curves which the path planner is unable to process.
Its overall form, comprised of occluded regions, is also beyond the scope of the intended application of the path planner developed.


\section{Results}
A test is considered a success if the path planning algorithm completes without error and a visual inspection confirms the paths planned.
If an error is thrown or if the paths planned deviate noticeably from the model's surfaces, the test is a failure.
Of the 18 class 1 models, paths were successfully planned on 12.
Paths were unable to be planned on the remaining models.
The paths planned from 2 of the 12 successes are shown in Figure~\ref{fig:class_1_results}.

\begin{figure}[htb]
	\centering
	\begin{subfigure}{0.4\textwidth}
		\centering
		\def\modelName{91142}
\begin{tikzpicture}[scale=2.5]
	\begin{axis}[no axes, cycle list name=bous-path-3D]
		\foreach \f in {0,...,13} {
			\edef\FileName{R\TwoDigits{\f}wpts.obj}
			\addplot3 table[x index=1, y index=2, z index=3] {../resources/models/m\modelName/\FileName};
		}
	\end{axis}
\end{tikzpicture}
		\caption{Model \modelName{} paths}
		% \label{sfig:91142_paths}
	\end{subfigure}
	\hfill
	\begin{subfigure}{0.4\textwidth}
		\centering
		\def\modelName{1505020}
\begin{tikzpicture}[scale=2.4]
	\begin{axis}[no axes, cycle list name=bous-path-3D,view={100}{-40}]
		\foreach \f in {0,...,31} {
			\edef\FileName{R\TwoDigits{\f}wpts.obj}
			\addplot3 table[x index=1, y index=2, z index=3] {../resources/models/m\modelName/\FileName};
		}
	\end{axis}
\end{tikzpicture}
		\caption{Model \modelName{} paths}
		\label{sfig:1505020_paths}
	\end{subfigure}
	\caption{Paths planned for two class 1 models}
	\label{fig:class_1_results}
\end{figure}

Close inspection of the paths shown in Figure~\ref{fig:class_1_results} reveals that the paths are disconnected from one another.
This is because the TSP solver was cut from the scope of this project.
Due to the octagonal hole in model 1505020 (see Figure~\ref{sfig:1505020_paths}) the top and bottom faces were segmented into 8 sub-regions during the \textit{Interior Edge Extension} step.
The full list of models, outcomes, and a brief explanation of the cause of failure can be found in Appendix~\ref{app:model_table}.



%! TeX root = thesis.tex
\chapter{Discussion}
\IMRADlabel{discussion}
Discussion! did it work? I hope so.



\chapter{Conclusion}
% This is what it /can/ do well, and /this/ is where it falls short.
This project, or at a minimum the overall concept chosen, proved more arduous than anticipated and requiring more time than available.
The complete pipeline is capable of processing models with planar surfaces.
At the time of writing, corners where four or more edges converge cause issues, and should be avoided if possible.
(? Given a few extra days, fixing this should be easily possible.)
As for curved surfaces, concave curves will at a minimum cause oversegmentation during the \textit{2D Segmentation} step, as discussed in the previous chapter.
The procedure is also unable to properly handle convex curves, but would require less effort to finish that functionality.
Furthermore, smooth transitions between primitive types constitute another open challenge.

Throughout development great emphasis was placed on reaching a point where the program and procedure could be tested in its entirety, and imroved from there.
In hindsight, this approach to development was ill-suited to a project of this nature, where the main building blocks are each complex and somewhat fragile.
If the time had been taken to test and refine \textit{Watershed Segmentation} and \textit{Surface Classification} to maturity, the issue regarding the height function selection could have been detected early enough to solve within the bounds of this project.

% Current conclusion(s):
% \begin{enumerate}
% 	\item I did not do sufficient research in the beginning, leading to wasted efforts later on
% 	\item My workflow was short-sighted, in that i only thought / considered a few steps ahead of my current position, rather than having a solid "big picture" view
% 	\item For simple flat objects with sharp edges, it works fine
% 	\item ...?
% \end{enumerate}

% I suggest /not/ continuing without serious reconsideration of either the objectives or the overall concept.
A continuation of this project without serious reevaluation of the overall concept is strongly disadvised.
The geometric primitive segmentation and classification concept is likely to be too rigid to effectively handle truly ``arbitrary'' geometry.
Should the range of the project's input geometry be restricted in future iterations, then this concept may prove viable.

\section{Outlook}
There are plenty of aspects of this project that could be improved.
These range from mere bug-fixes to replacing entire algorithms.
If the project were to be continued with all high-level components and steps, \textit{Interior Edge Extension} would require more testing to improve its robustness and fix minor bugs.

In order to maintain the current concept while allowing high-level and algorithm changes, the following improvements are recommended.

Further work would include:
\begin{itemize}
	\item An approach to handle both ``hard'' \textit{and} ``soft'' edges must be developed.
	\item Implement regression and classification functions for remaining geometric primitives
	\item \textit{Geometry Simplification} should be expanded and improved to solve the aforementioned shortcomings, and to improve robustness.
	\item The \textit{Interior Edge Extension} step should be replaced with one better suited to shapes with curved edges.
		The Boustrophedon Cellular Decomposition or an expansion thereof would provide a decent starting point for its replacement.
	\item The Traveling Salesman Problem should be implemented to join the individual cellular paths.
	% \item Inverse kinematics that better take into account the rotary table
	% \item Reachability was basically ignored in this project, so that should be explored...
	\item Reachability of the poses generated by the path planner will need to be taken into account before the procedure may be used outside of a test environment.
\end{itemize}



\appendix

%! TeX root = thesis.tex
\chapter{Test Objects}\label{app:model_table}

The following are the 3D models used to test the program.
The values listed under \textbf{Source} have the following meanings:
\begin{itemize}
	\item \textbf{GC-PF}: Model downloaded from GrabCAD created by user paulocferreira 3D\cite{GC-PF}.
		User paulocferreira 3D numbers their basic models, which makes finding the source for a given model fairly straightforward.
		A specific model can be found by finding the model that corresponds to the number from the ID of a \textbf{GC-PF} sourced model: \verb|fcXXX|.
	% \item \textbf{GC-[A]}: Model was downloaded from GrabCAD, created by user XXX
	\item \textbf{10K}: Model downloaded from Thinki10K, a database of 3D models\cite{Thingi10K_paper, Thingi10K_app}.
		The models in Thingi10K were collected originally from Thingiverse, hence the name.
		To find the specific webpage for a given model, replace \verb|XXX| in \verb|https://ten-thousand-models.appspot.com/detail.html?file_id=XXX| with the ID from the table.
	\item \textbf{Thing}: Model downloaded from Thingiverse.
		Specific Thingiverse webpages can be found by replacing the \verb|XXX| in \verb|https://www.thingiverse.com/thing:XXX| with the ID from the table.
\end{itemize}

% The causes of failure generally fall into one of five categories.
% An elaboration of each \textbf{Explanation} message is found in the following:
The notes in the \textbf{Explanation} column expand to the following meanings:
\begin{itemize}
	\item Mesh increase: A mesh increase enabled the path planning success
	\item Disabled corner check: Path generation was able to proceed after disabling the duplicate corner check.
	\item Watershed error: Watershed repeatedly produced a composite mesh section, even as the merge threshold approached 0, causing an error.
	\item Narrow region: A narrow region was determined to be the root cause of a Watershed error
	\item Invalid corner: A corner was created with only two adjacent regions.
	\item Cylinder: Model contains a cylinder or other complete round edge, which is unable to be processed by \textit{Geometry Simplification}
	\item Soft edge: The model contains a smooth transition from one primitive type to another.
	\item Out-of-scope geometry: The model contains geometry that is beyond the scope of this project, such as swept curves.
\end{itemize}

\vspace{5mm}
\csvreader[%
	longtable = |lcccll|,
	respect underscore = true,
	table head = \hline
	\textbf{ID} &
	\textbf{Source} &
	\textbf{Mesh} &
	\textbf{Class} &
	\textbf{Outcome} &
	\textbf{Explanation}\\\hline\hline,
	late after line = \\\hline,
	late after last line = \\\hline
]{../resources/model_table.csv}{
	Filename=\id,
	Source=\src,
	Mesh=\mesh,
	Class=\class,
	Outcome=\outcome,
	Explanation=\explanation
} {% Comment to prevent first couple lines from being shifted right slightly
	\id & \src & \mesh & \class & \outcome & \explanation
}% https://tex.stackexchange.com/a/619232



\printbibliography

\end{document}
