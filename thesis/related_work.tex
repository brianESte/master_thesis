%! TeX root = thesis.tex
\chapter{Related Work}\label{related_work}

Within the topic of "path planning" there exists "pathfinding" and "coverage path planning".
The goal of pathfinding is to determine the shortest path between two points while avoiding any obstacles.
Pathfinding has applications ranging from navigation to network routing.
The primary pathfinding algorithms are: Dijkstra's algorithm\cite{Dijkstra, Improved_Dijkstra} which determines the shortest path between two nodes in a weighted graph, and A* (A-Star)\cite{A_Star_lit_review, A_Star_beginners, A_Star_in_computer_games} which is effectively Dijkstra's algorithm with a heuristic function applied.
Dijkstra's algorithm is well suited for navigating between cities\cite{Dijkstra_for_railroads}, as a network of connected cities is effectively a graph with edges weighted by distance.
(maybe talk about A* a little more?)


In contrast to this, the goal of coverage path planning (CPP) is to traverse as much of the given working area as possible.
The objective of applied CPP is often for an agent to perform some function over the entire working area.
Examples of this include path planning of farmland\cite{CPP_3D_farming_terrain, CPP_alg_agriculture}
of CPP is to
