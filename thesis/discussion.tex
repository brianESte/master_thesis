%! TeX root = thesis.tex
\chapter{Discussion}
\IMRADlabel{discussion}
Discussion! did it work?

\section{Watershed Segmentation}
Watershed segmentation as implemented here, functioned as intended.
Based on the results observed, the cause of any segmentation issues is believed to lie with the height function chosen.
Though not thoroughly tested, the Watershed modifications made in this work are expected to have been non-detrimental.
(? probably better without the previous sentence...)
% There is no reason to believe the additions / modifications / implementation here is responsible for the issues encountered.

\section{Surface Classification}
Of the implemented primitives, plane and elliptic cylinder, surface classification works as intended.
Expanding this project to handle the third proposed primitive, ellipsoids, would be relatively straightforward, given that Eberly's work on ellipses extends to ellipsoids \cite{GeoTools_pt_to_ellipse}.
Should the approach of primitive segmentation and classification be pursued further, the functionality to handle more primitives will be necessary.
Beyond ellipsoids, the main remaining geometric primitives are the cone and torus.
For both of these Eberly once again offers solutions \cite{GeoTools_least_squares_fitting}, though fitting a cone to 3D points can likely be simplified by exploiting the input mesh's normals.
% The surface classification methods presented here, although they represent earnest attempts, are generally unreliable. -> less so since classifiers were replaced with pure regression.

\section{2D Segmentation}
The stated goals of the \textit{interior edge extension}\cite{IntEdgeExt} are:
\begin{enumerate}
	\item to minimize the number of changes in direction of the planned paths.
		This goal leads to the optimization requirement of minimizing the sum of cell widths when solving the weighted set coverage problem.
	\item to produce convex cells.
\end{enumerate}

Minimizing the number of direction changes is of value if the cost in time or energy of a direction change is non-negligible, as is the case for mobile robots such as automated agriculture machines, AUVs, and to a lesser degree UAVs.
However the cost of direction change for a robot arm is minimal.
Furthermore, the envisioned end-effector tool is equally effective in its surface treatment moving in the positive or negative direction normal to the laser plane.
(? rethink previous sentence...?)
That is, while a Roomba would be less effective if it were to traverse its environment backwards, the laser module suffers no such loss of efficacy.
% a ``planar laser'' (? revisit with better wording)
% need not rotate at each switchback because it is
Hence, the goal of minimzing the number of switchbacks was unnecessary for this work.

Nielsen et al. do not explain why the final polygons should be convex, treating it as a given.
Boustrophedon path planning requires cells to be \textit{semi}-convex, not necessarily truly convex.
This is illustrated in figure \ref{fig:bpath_semi_convex} below.

\begin{figure}[htb]
	\centering
	\begin{tikzpicture}[scale=1.0]
		\datavisualization[
			% new Cartesian axis=x axis, x axis={attribute=x},
			% new Cartesian axis=y axis, y axis={attribute=y},
			scientific axes,%=clean,
			all axes={ticks={major={at={}},minor={at={}}}},
			data/format=table,
			% yticklabel={\empty},
			% school book axes,
			visualize as line/.list={outline,path},
			% /data point/set/outline/.initial=1
			% outline={style={}},
			path={style={red}}
			]
			data[
				headline={x,y},
				read from file="../resources/boustrophedon/ex_2_polygon.csv",
				set=outline]
			data[
				headline={x,y},
				read from file="../resources/boustrophedon/ex_2_path.csv",
				set=path];
	\end{tikzpicture}
	\caption{Boustrophedon path planning performed on a semi-convex shape}
	\label{fig:bpath_semi_convex}
\end{figure}

For semi-convex polygons the sweep direction may no longer be chosen based on the longest edge.
The sweep direction must be chosen such that a line parallel to the sweep direction reaches the entire polygon when swept along the shift direction.
% A formal definition of ``semi-convex'' is beyond the scope of this work, and left for future work, but ..?
Although a formal definition of ``semi-convex'' is beyong the scope of this work, an informal one may be attempted (? approached).
% This is best described with an image, see below.

\subsection{Isovist Line}
% TODO: new name... "bivist line" ?
To define a useful measure of "semi-convexity" the notion of the isovist is helpful.
Benedikt gives the definition of an isovist as "the set of all points visible from a given vantage point in space and with respect to an environment"\cite{Isovists}.
While the isovist concept is primarily used in architecture and urban spaces, it can also be applied to computational geometry.
In the \textit{art gallery problem} each guard represents an isovist, and the solution to a given art gallery problem is the minimum number of isovists to completely cover the given environment.
A similar problem in computational geometry is the \textit{watchman route problem}, in which the a single mobile watchman is tasked with guarding an environment with obstacles.
The challenge lies in determining the shortest path the watchman should take to observe the entire environment.
Determining a polygon's ``semi-convexity'' draws on both of these problems by reducing the number of guards is to one, whose scope of observation is further reduced to a single line in both directions, but who \textit{is} allowed to move perpendicular to the viewing direction.
% Extending the art gallery problem to create a semi-convexity-test for a polygon, the number of guards is reduced to one, whose scope of observation is further reduced to a single line in both directions, but who \texti{is} allowed to move perpendicular to the viewing direction.
If ``guarding'' the polygon this way is feasible, then the polygon is at least semi-convex, and boustrophedon path planning may be applied, with the shift direction set to the guard's direction of movement, and the sweep direction perpendicular thereto.
Naturally, convex polygons pass this test as well.
% Each guard is reduced to looking along a single line in both directions, but is allowed to move along a line perpendicular to the viewing direction.

\section{Cellular Path Planning}
Cellular path planning using boustrophedon path planning worked as expected.
There was potential for expansion, such as allowing the TSP to potentially determine the path start point on the polygon.
This is left for a future expansion of the project, in which the TSP is fully implemented.

