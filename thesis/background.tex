%! TeX root = thesis.tex
\chapter{Background}
Here i will discuss background knowledge of Coverage Path Planning and surface curvature...

Things / topics i will cover:
\begin{enumerate}
	\item watershed segmentation... -> background, methodology
	\item geometric curvature -> background
	\item UV Mapping -> background
	\item Geometry simplification -> methodology?
	\item Interior Edge Extension (and my modifications, if any) -> methodology
	\item boustrophedon -> methodology
	\item Modified TSP -> methodology
\end{enumerate}


\section{Geometric Curvature}
For segmenting a 3D mesh via Watershed Segmentation each vertex requires a singualr / combined measure of curvature.
Various forms and types of curvature were examined throughout this project and are presented in the following sections.

\subsubsection{Principal Curvatures}
All combined measures of curvature are based, either in theory or in actuality, on the principal curvatures.
The principal curvatures at a given point on a surface are simply the maximum and minimum curvature.
Determining principal curvatures for a continuous surface is relatively clear and unambiguous.TODO: insert citation describing principal curvature on continuous surface
NOTE: maybe also include a graphic showing principal curvatures ?
There are however various ways of approximating the principal curvatures on a discrete mesh.TODO: insert citations of principal curvature comparisons

\subsubsection{Mean Curvature}
Mean curvature is, as the name implies, the mean of the principal curvatures.
TODO: insert basic graphic re mean curvature
TODO: talk about pros(?) of mean curvature
Mean curvature however encodes little information and was not seriously considered by itself in this work.
Mean curvature can be approximated on a discrete mesh in various ways...TODO

\subsubsection{Gaussian Curvature}
Gaussian curvature, (where does the name come from?) is the product of the principal curvatures.
Gaussian curvature has the benefit of being more sensitive to changes in each of the principal curvatures, but it cannot distinguish "developable" surfaces, such as cylinders and planes.
TODO: Insert picture visually detailing gaussian curvature
Gaussian curvature can be approximated on a discrete mesh in various ways...TODO

\subsubsection{Root Mean Square Curvature}


\subsubsection{Derivative of Curvature}

\section{UV Mapping}
UV Mapping is the process of projecting a surface from 3D to 2D, effectively "unwrapping" the 3D surface.
(insert sentence about common usage in 3D modeling / graphics and textures)
\subsection{Conic Surfaces}
General equation of a conic is:
\begin{equation}
	Ax^2 + Bxy + Cy^2 + Dx + Ey + F = 0
\end{equation}

\subsubsection{Elliptic Surface}
General equation:
\begin{equation}
	\frac{x^2}{a^2} + \frac{y^2}{b^2} = 1
\end{equation}
To obtain a vector perpendicular to a point on the ellipse, the derivative of said point is rotated 90 degrees.
The point, as a function of the angle $\theta$:
\begin{equation}
	(x,y) = (a\cos\theta, b\sin\theta)
\end{equation}
The derivative thereof:
\begin{equation}
	\frac{d}{d\theta}(x,y) = (-a\sin\theta, b\cos\theta)
\end{equation}
Rotated -90 degrees:
\begin{equation}
	\text{Rot}_{90}\frac{d}{d\theta}(x,y) = (b\cos\theta, a\sin\theta)
\end{equation}

\subsubsection{Parabolic Surface}
Parabolas are much simpler than and ellipses and hyperbolas, as can be seen by the following equations.
The parametric general equation of a parabola is:
\begin{equation}
	-\sin\theta x + \cos\theta y = \frac{1}{4f}(\cos\theta x + \sin\theta y - h)^2 + k
\end{equation}
To solve for the conic parameters of a parabola the general equation's quadratic term is expanded and like terms gathered:
\begin{multline*}
	-\sin\theta x + \cos\theta y = \frac{1}{4f}(\cos^2\theta x^2 + \cos\theta\sin\theta xy \\
	- h \cos\theta x + \cos\theta\sin\theta xy + \sin^2\theta y^2 - h\sin\theta y - h \cos\theta x - h \sin\theta y + h^2) + k
\end{multline*}
\begin{multline*}
	\frac{\cos^2\theta}{4f} x^2 + \frac{2\cos\theta\sin\theta}{4f} xy + \frac{\sin^2\theta}{4f} y^2 \\
	+ \left(\sin\theta - \frac{2h \cos\theta}{4f}\right)x + \left(\cos\theta - \frac{2h \sin\theta}{4f}\right)y + \frac{h^2}{4f} + k = 0
\end{multline*}
\begin{align}
	A &= \frac{\cos^2\theta}{4f} \\
	B &= \frac{2\cos\theta\sin\theta}{4f} \\
	C &= \frac{\sin^2\theta}{4f} \\
	D &= \sin\theta - \frac{2h \cos\theta}{4f} \\
	E &= \cos\theta - \frac{2h \sin\theta}{4f} \\
	F &= \frac{h^2}{4f} + k
\end{align}

\subsubsection{Hyperbolic Surface}
\begin{equation}
\begin{split}
	\frac{(\cos\theta(x-h) + \sin\theta(y-k))^2}{a^2} - \frac{(\cos\theta(y-k) + \sin\theta(x-h))^2}{b^2} &= 1 \\
	\frac{(x_c + y_s)^2}{a^2} - \frac{(y_c + x_s)^2}{b^2} &= 1 \\
	b^2(x_c^2 + 2x_c y_s + y_s^2) - a^2(y_c^2 + 2 x_s y_c + x_s^2) &= a^2 b^2 \\
	b^2 x_c^2 + 2 b^2 x_c y_s + b^2 y_s^2 - a^2 y_c^2 - 2 a^2 x_s y_c - a^2 x_s^2 - a^2 b^2 &= 0 \\
	b^2 x_c^2 - a^2 x_s^2 + 2 b^2 x_c y_s - 2 a^2 x_s y_c + b^2 y_s^2 - a^2 y_c^2 - a^2 b^2 &= 0 \\
\end{split}
\end{equation}
\begin{multline*}
	b^2 (\cos\theta(x-h))^2 - a^2 (\sin\theta(x-h))^2 \\
	+ 2 b^2 \cos\theta(x-h) \sin\theta(y-k) - 2 a^2 \sin\theta(x-h) \cos\theta(y-k) \\
	+ b^2 (\sin\theta(y-k))^2 - a^2 (\cos\theta(y-k))^2 - a^2 b^2 = 0
\end{multline*}
\begin{multline*}
	(b^2 \cos^2\theta - a^2 \sin^2\theta)(x-h)^2 \\
	+ 2 \cos\theta\sin\theta(b^2 - a^2)(x-h)(y-k) \\
	+ (b^2 \sin^2\theta - a^2 \cos^2\theta)(y-k)^2 - a^2 b^2 = 0
\end{multline*}
\begin{equation*}
	\begin{split}
		c_1(x-h)^2 + c_2(x-h)(y-k) + c_3(y-k)^2 - a^2 b^2 &= 0 \\
		c_1(x^2-2hx + h^2) + c_2(xy-hy-kx+hk) + c_3(y^2-2ky+k^2) - a^2 b^2 &= 0 \\
		c_1 x^2 + c_2 xy + c_3 y^2 + (-2h c_1 -k c_2) x + (-h c_2 -2k c_3)y + c_1 h^2 + c_2 hk + c_3 k^2 - a^2 b^2 &= 0 \\
	\end{split}
\end{equation*}
\begin{align*}
	A &= c_1 = b^2 \cos^2\theta - a^2 \sin^2\theta \\
	B &= c_2 = 2 \cos\theta\sin\theta(b^2 - a^2) \\
	C &= c_3 = b^2 \sin^2\theta - a^2 \cos^2\theta \\
	D &= (-2h c_1 -k c_2) \\
	E &= (-h c_2 -2k c_3) \\
	F &= c_1 h^2 + c_2 hk + c_3 k^2 - a^2 b^2
\end{align*}
Now to solve for $a$, $b$, and $\theta$:
Solving $A$ for $b^2$:
\begin{equation}
	\begin{split}
		A &= b^2 \cos^2\theta - a^2 \sin^2\theta \\
		b^2 &= \frac{A}{\cos^2\theta} + a^2\tan^2\theta \\
	\end{split}
\end{equation}
Setting 3.6 into $C$ and solving for $a^2$:
\begin{equation}
	\begin{split}
		C &= b^2 \sin^2\theta - a^2 \cos^2\theta \\
		a^2 &= -\frac{C}{\cos^2\theta} + b^2\tan^2\theta \\
		a^2 &= -\frac{C}{\cos^2\theta} + \left(\frac{A}{\cos^2\theta} + a^2\tan^2\theta\right)\tan^2\theta \\
		\cos^4\theta a^2 &= -C\cos^2\theta + A\cos^2\theta + \sin^4\theta a^2 \\
		(\cos^4\theta - \sin^4\theta) a^2  &= \cos^2\theta(A-C) \\
		a^2  &= \frac{\cos^2\theta(A-C)}{(\cos^4\theta - \sin^4\theta)} \\
	\end{split}
\end{equation}
Setting 3.6 into $B$:
\begin{equation}
	\begin{split}
		B &= 2 \cos\theta\sin\theta(\frac{A}{\cos^2\theta} + a^2\tan^2\theta - a^2) \\
		\cos^2\theta B &= 2 \cos\theta\sin\theta(A + (\sin^2\theta - \cos^2\theta) a^2) \\
	\end{split}
\end{equation}
Setting 3.7 into 3.8:
\begin{equation}
	\begin{split}
		\cos^2\theta B &= 2 \cos\theta\sin\theta(A + (\sin^2\theta - \cos^2\theta) \frac{\cos^2\theta(A-C)}{(\cos^4\theta - \sin^4\theta)}) \\
	\end{split}
\end{equation}


