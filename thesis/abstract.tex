%! TeX root = thesis.tex
\section*{Abstract}
The handling and processing of hazardous waste poses unnecessary risks to humans in the modern age.
In order to protect metal components from corrosion and facilitate decomissioning, nuclear power plants employ a protective coating on parts that are expected to come into contact with radioactive nuclides.
During dismantling the protective coating is stripped from those parts so that the material underneath may be processed normally.
% The processing of irradiated material, specifically from the dismantling of nuclear power plants, requires the removal of the protective coating therefrom.
% To minimize human contact with hazardous waste, the processing thereof must be automated.
% The complete removal of this protective coating is an application of coverage path planning.
The pieces removed during the dismantling process can take any form, complicating the task of removing their coating.
% In this project a coverage path planning method based on segmentation and simplification is presented.
This project presents a coverage path planning method based on segmentation and simplification with the aim of automating the waste cleaning process.

Watershed segmentation is harnessed for the 3D segmentation step to divide the mesh along regions of high curvature.
The resultant mesh sections are classified as geometric primitives, and subsequently simplified down to those forms.
The problem is reduced to 2D by unwrapping these simplified surfaces.
Non-convex regions are further decomposed to achieve convexity in order to facilitate local path planning.
Upon each convex region a boustrophedon path is planned.
The individual paths are joined into one, with the traversal order determined via a Traveling Salesman Problem in order to minimize the overall path length.
With the path complete, it is send via XMLRPC to the robotic arm tasked with the path's execution.

%\clearpage
% \section*{Kurzfassung}
% Auch auf Deutsch?
