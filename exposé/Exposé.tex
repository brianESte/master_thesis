\documentclass[a4paper, 10pt]{article}

\usepackage[T1]{fontenc}
\usepackage[utf8]{inputenc}

\usepackage[a4paper, total={160mm,230mm}]{geometry}
\usepackage{fancyhdr}
\usepackage{graphicx}

\usepackage[english]{babel}

% Bibliography
\usepackage[
%	backend=biber,
	backend=bibtex,
	style=numeric
]{biblatex}
\addbibresource{references.bib}
%\bibliographystyle{plain}
\usepackage{csquotes}

\graphicspath{ {./images/} }

\pagestyle{fancy}
\fancyhf{}
\rhead{\today}
\chead{Exposé}
\lhead{Brian Stephenson}

\fancyfoot[C]{\thepage}
\begin{document}
	
\begin{center}
	\fontsize{24pt}{10pt}\selectfont
	\textsc{\textbf{Coverage Path Planning on Arbitrary Geometry}}
\end{center}
\begin{center}
	15.03.2025 - 15.09.2025 \\
	Brian Stephenson, Matrikel-Nr: 2924906
\end{center}

\section*{Motivation}
% Why cpp an object of arbitrary geometry?
Processing of irradiated or otherwise toxic waste poses unnecessary risks to humans in the modern age.
With the advent of 3D scanning technology and autonomous robots, humans need no longer manually handle harmful waste.
In such a system, human contact is limited to placing the waste object in the system, repositioning the object as needed, and removing the cleaned object at the end.

The intended application is the processing of waste and debris from the dismantling of nuclear power plants, but other applications are possible.
The primary cooling system in a nuclear power plant is irradiated throughout its lifespan, thus components thereof are treated as hazardous waste.
Coatings are used at nuclear power plants to protect metals from corrosion, reduce general wear, and facilitate decontamination of containment walls and surfaces \cite{NRC_coatings}.
Because the coating exists to block radioactive nuclides from penetrating the metal underneath, once removed, the metal can be treated like ordinary material.
This greatly reduces the volume of material that requires special treatment and storage.

%Such a system already exists and is in operation in Biblis, Germany, but uses a (supposedly) inefficient algorithm and a pressure washer for coating removal.
The updated(?) system will use a high power laser to ablate the coating, reducing necessary clean-up post processing.

The difficulties in an autonomous waste processing system lie in handling arbitrary geometry, efficient computing of the geometry and path, as well as ensuring the entire surface is processed.
To efficiently cover the given object's surfaces, it is proposed to segment the surface geometry into planar, cylindric, and convex primitive regions.
Watershed segmentation was chosen as the basis for the partitioning algorithm.
Watershed Segmentation applies a "height" function to each vertex in a mesh, and partitions the mesh into low regions with high bounds \cite{Watershed, Hier_seg_autobody_painting}.
Concave regions are further partitioned via ??? to produce compact convex regions.

Testing is currently conducted using a UR10 from Universal Robots.
During prior testing, commands were sent to the robot without being certain that the previous command was executed fully.
Newly received commands would interrupt the current command, leading to jerky, and sometimes erroneous movement.
In order to prevent such issues, XML-RPC was chosen as the communication method between controller and robot for this project \cite{UR_XML-RPC}.
XML-RPC (Remote Procedure Call) is a communication protocol that allows programs to communicate via HTTP independent of OS and architecture.
A cursory examination of the UR programming facilities revealed nothing akin to a queue with which to hold received commands.
As such, queuing of commands will take place server-side (controller) and the robot will be programmed to process commands individually, requesting the next as the current is being completed.

\begin{flushleft}
	\section*{Goal}
	The goal of this master's thesis is to develop a coverage path planning algorithm able to handle objects of arbitrary geometry.
	The algorithm should cover as much of the target surface as is possible.
	Communication between the path planner and the robot should be seamless, such that the robot's movements are uninterrupted by path planner communication.
\end{flushleft}
\begin{flushleft}
	\section*{Task}
	\begin{enumerate}
		\item Develop coverage path planning algorithm
		\begin{enumerate}
			\item Develop and implement robust surface segmentation algorithm
			\item Develop or obtain an inverse-kinematic solver to ensure clean robotic movement
			\item Evaluate algorithm using 3D printed test objects
		\end{enumerate}
		\item Communicate with robot via XML-RPC
		\begin{enumerate}
			\item Create a client-side (robot) program to accept and execute commands via XML-RPC
			\item Test connection to a virtual machine
			\item Test connection to a demonstration robot
		\end{enumerate}
		\item Test various sample objects to validate algorithm robustness
		\item Test algorithm and connection on UR3
	\end{enumerate}
\end{flushleft}
%\newpage
\printbibliography
\end{document}